\Exhibit{Resume}{Резюме Алексея Инкина}

\UnnumberedTitle{Алексей Инкин}

\begin{tabular}{ll}
    Email: & <\dots>\\
    Телефон: & <\dots>\\
    Telegram: & <\dots>\\
    Адрес: & <\dots>, Batumi, 6000, Georgia\\
    LinkedIn: & https://www.linkedin.com/in/alexey-inkin-784b7371/\\
    GitHub: & https://github.com/alexeyinkin\\
    Medium: & https://medium.com/@alexey.inkin\\
    Instagram: & https://www.instagram.com/alexey.inkin/\\
    VK: & https://vk.com/ainkin\\
\end{tabular}

\UnnumberedTitle{Навыкм}

\begin{itemize}
    \item Программирование: Dart, Flutter, PHP, JavaScript, TypeScript, C++, Java, SQL.
    \item Технологии: Git, GitHub Workflows, RabbitMQ, Google Cloud, Gradle, HTML, CSS.
    \item Менеджмент: Software project management, Team leadership.
    \item Языки: English (fluent), Russian (native).
\end{itemize}

\UnnumberedTitle{Членство}

\begin{tabular}{rl}
    Июль 2023 -- & Google Developer Experts, Dart \& Flutter Expert\\
    Июль 2023 -- & \Ieee (старший член с сентября 2023)
\end{tabular}

\UnnumberedTitle{Опыт работы}

\ResumeItem{Lead Flutter Developer}{Akvelon}{июнь 2022 -- июнь 2023, Тбилиси, Грузия, удалённо}{%
Вёл команду фронтенд-разработки для проектов Google и Apache.
Полностью спроектировал и управлял реализацией:

— Apache Beam Playground, фронт, https://play.beam.apache.org\\
— Tour of Beam, фронт, https://tour.beam.apache.org\\
— Flutter Code Editor со сворачиванием кода и т.п, https://pub.dev/packages/flutter\_code\_editor\\
— Портировал HighlightJS на Dart, https://pub.dev/packages/highlighting

Обязанности:

— Планирование этапов.\\
— Дизайн ахритектуры.\\
— Разработка редактора кода и LMS.\\
— Защита решений перед инженерами Google, которые разрабатывают Apache Beam.\\
— Ревью кода.\\
— Собеседование кандидатов в команду.\\
— Менторство разработчиков.\\
— Презентация приложений на конференции.
}

\ResumeItem{Flutter Team Leader}{Boss of Life}{май 2021 -- февраль 2022, Лондон, Великобритания, удалённо}{%
— Менторство разработчиков.\\
— Реализация $\sim$40 экранов мобильного приложения.\\
— Разработка структуры БД Firebase.\\
— Реализация бэкенда из $\sim$100 Firebase Cloud Functions.\\
— Аудит безопасности.\\
— Управление приёмочным тестированием.
}

\ResumeItem{Senior PHP Developer}{Calltouch}{июль 2019 -- сентябрь 2020, Москва, Россия, офис}{%
— Рефакторинг системы аутентификации, поиск и исправление проблем безопасности.\\
— Реализовал генерацию отчётов по рекламе с многочисленными метриками из внешних платформ, маппинг больших объёмов данных.\\
— Сделал в Telegram Messenger.\\
— Настроил статического анализа кода с помощью PHPStan.\\
— Повысил покрытия тестами с Codeception.

Технологии включали PHP 7.1, Symfony 3, PostgreSQL 9.6, RabbitMQ, Redis, Docker, GitLab, Jira, Confluence, Crucible.

Calltouch -- лидер кол-трекинга в России с выкоконагруженными серверами.
}

\ResumeItem{Senior PHP Developer}{Фриланс}{январь 2017 -- июль 2019, Россия}{%
Собственная система управления обучением и движок электронной коммерции.
Технологии:

— PHP 7.3, JavaScript, HTML5, CSS3, jQuery, MySQL.\\
— Continuous integration: Git, PHPStan, PHPUnit.\\
— Java Swing и ffmpeg для сжатия и загрузки видеоукроков.\\
— Facebook API и VK API для авторизации.\\
— Google Maps API для просмотра клиентов на карте.\\
— UniSender для email-маркетинга.

Система включала:

— Цепочки видеоуроков, различные уровни доступа для студентов, автоматический и ручной зачёт заданий и доступ.\\
— Защита видео от скачивания всеми браузерными плагинами через HLS-шифрование и одноразовые ссылки на
сегменты видео.\\
— Интернет-магазин с интеграцией с основными платёжными шлюзами (Яндекс, Tinkoff).\\
— Движок партнёрской программы со счётчиком Google Analytics для каждого партнёра.
}

\ResumeItem{Chief Technology Officer}{Тренинговый центр Сергея Граня}{ноябрь 2014 -- апрель 2016, Санкт-Петербург и Казань, Россия, офис}{%
В качестве CTO:

— Разработал учебную платформу.\\
— Интегрировал сторонний движок партнёрской программы.\\
— Разработал автоматизированную воронку, через которую наняли 100+ преподавателей на удалённую работу по совместительству.

В качестве руководителя выпуском материалов:

— Нанял дизайнеров, иллюстраторов, редакторов видео, транскрибаторов и корректоров.\\
— Выпустил более 100 часов видео от записи с камеры и рукописей до полноценных анимированных слайд-шоу.\\
— Написал 150 страниц упражнений для учеников: стратегия запуска бизнеса, лайф-менеджмент, планирование рекламы и др.\\
— Разработал рекламные материалы с высокой конверсией.

Доход на пике -- \$750'000 в 2015, из которых \$350'000 -- чистая прибыль владельца.
Это в 50 раз больше среднего дохода в России.

Тренинговый центр Сергея Граня продавал видеокурсы по бизнесу.
У них было 2500 клиентов
и продукты от \$100 до \$1500.

Технологии: PHP 5, MySQL, CSS3, jQuery.
}

\ResumeItem{Основатель}{GetSoft.ru}{декабрь 2003 -- март 2014}{%
GetSoft.ru объединил больше 3000 авторов софта с клиентами до того, как появился Windows
Market.
Проект был бесплатным для всех, и я зарабатывал на рекламе.
Технологии: PHP 4, JavaScript, CSS2, MySQL.
}

%\ResumeItem{PHP Developer}{}{2009--2013, Nizhny Novgorod, Russia}{%
%Short projects for gyms, dentists, fire safety, factories, and dozens of other small businesses.
%}

\ResumeItem{.NET Developer}{Telma}{февраль 2005 -- март 2006, офис, частичная занятость}{%
Доработал внутреннюю систему учёта контрактов и внутренний форум сотрудников.
}

\UnnumberedTitle{Награды}

\begin{tabular}{r|l}
    2023 & Google Developer Expert\\
    2021 & Всероссийский хакатон `Медицина, Здравоохранение и Наука',\\
    & `Цифровой Прорыв', Россия, 3-е место в кейсе iHerb\\
\end{tabular}

\UnnumberedTitle{Образование}

\ResumeItem{Инженер, Информационные системы и технологии}{Нижегородский Государственный Технический университет}{сентябрь 2002 -- январь 2008, Нижний Новгород, Россия}{%
}

\UnnumberedTitle{Волонтёрство}

\ResumeItem{Senior Member Review Panel Member}{\Ieee}{с ноября 2023}{}

\ResumeItem{Наблюдатель на выборах Президента}{`Голос', Штаб Прохорова, Штаб Явлинского}{2012, 2018, Нижний Новгород, Россия}{}

\pagebreak
