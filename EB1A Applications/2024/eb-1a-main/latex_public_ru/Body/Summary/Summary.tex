\section{Обзор достижений и квалификации Алексея Инкина}

Алексей Инкин -- инженер, специализирующийся на разработке ПО, см. его резюме в \ExhibitRef{Resume}.
Он окончил Нижегородский Государственный Технический университет с отличием.
У него образование бакалавра \ExhibitRef{Diploma1}
и инженера информационных технологий \ExhibitRef{Diploma2}.

За 19 лет в профессии он работал в областях разработки бэкэнда на языке PHP
и фронтенда с использованием Flutter, технологии от Google.

Алексей Инкин награждён званием `Google Developer Expert',
престижной наградой от Google \SectionRef{subsubsec:AwardGde}.
У него также есть призовое место на хакатоне `Цифровой Прорыв',
крупном и национально признанном в России \SectionRef{subsubsec:AwardLod}.

Алексей Инкин -- старший член Института Инженеров Электрики и Электроники (IEEE),
крупнейшей профессиональной ассоциации для инженеров.
Звание старшего члена требует долгой значимой работы \SectionRef{subsubsec:AssociationsIeee}.
Алексей Инкин также член сообщества `Google Developer Experts'.
Это членство требует выдающихся достижений,
им обладают около тысячи человек в мире, и лишь около 100 из них специализируются на Flutter.
Это означает, что он входит в топ 0.025\% разработчиков, которые активно используют Flutter \SectionRef{subsubsec:AssociationsGde}.

Алексея Инкина приглашали оценивать достижения других кандидатов на звание старшего члена IEEE,
и он получил благодарственное письмо от председателя комитета,
который рассматривает заявления на получение этого звания \SectionRef{subsec:Judging}.

Алексей Инкин выполнял критическую роль в нескольких выдающихся организациях.
Он был разработчиком и руководителем команды фронтенда двух проектов для Apache Software Foundation,
одного из крупнейших в мире фондов open-source ПО
\SectionRef{subsubsec:RoleApache}.
Это было во время его работы в Akvelon, международной компании со штаб-квартирой в США.
Для Akvelon он также руководил разработкой специализированного редактора кода на Flutter,
который стал не только центральной частою тех двух проектов для Apache, но и самостоятельным ценным проуктом
\SectionRef{subsubsec:RoleAkvelon}.
Перед этим Алексей Инкин работал в Calltouch, крупной российской платформе аналитики рекламы.
Он отвечал за систему аутентификации пользователей, поиск и устранение
проблем с безопасностью, что является критическим по определению \SectionRef{subsubsec:RoleCalltouch}.
Он также был техническим директором в компании Сергея Граня, основателя Школы Граня,
где реализовал технические решения, которые увеличили доход компании в несколько раз \SectionRef{subsubsec:RoleGran}.

Алексей Инкин сделал значимый вклад в область разработки ПО.
Он спроектировал и реализовал значимый функционал в PHPStan, самом популярном статическом анализаторе
кода на языке PHP.
Этот функционал уменьшил ложные сообщения об ошибках в коде, которые выдаёт анализатор,
и упростил требования к коду для корректной работы анализатора.
Это произвело значительный эффект, поскольку PHP используется на большинстве сайтов в мире \SectionRef{subsubsec:ContributionsPhpStan}.
Кроме того, редактор кода на Flutter, разработанный Алексеем Инкиным и его командой,
стал вторым по популярности редактором для Flutter и используется в большом количестве приложений
\SectionRef{subsubsec:ContributionsFlutter}.

Алексей Инкин получал высокую зарплату во время работы в Akvelon,
в 3 раза выше средней в информационных технологиях в Грузии
и в 2.02 раза выше средней зарплаты программиста в Грузии \SectionRef{subsec:Salary}.

Алексей Инкин написал 49 технических статей в своём блоге на Medium.
9 из них были отобраны командой редакторов для широкого распространения \SectionRef{subsec:Articles}.

\pagebreak
