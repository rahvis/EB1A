\UnnumberedTitle{Петиция на резидентство по выдающимся способностям (EB-1A)}

\vspace{4em}

\begin{tabular}{ll}
    \textbf{Петиционер и иммигрант:} & \fl\\
    \textbf{Запрашиваемая классификация:} & Иммиграция по трудоустройству, первая категория\\
    & Выдающиеся способности в науке (EB-1A)\\
    & Раздел 203(b)(1) закона об иммиграции\\
    & и гражданстве [8 U.S.C. 1153]\\
    \textbf{Тип петиции:} & I-140
\end{tabular}

\vspace{2em}

Уважаемый офицер,

Нижеследующие доказательства направляются в поддержку петиции Алексея Инкина
для классификации как рабочего иммигранта первой категории
с выдающимися способностями в соответствии с разделом 203(b)(1)(A)
закона об иммиграции и гражданстве.

Эти доказательства демонстрируют, что Алексей Инкин обладает выдающимися способностями в науке,
в частности в \underline{разработке ПО},
что он обладает устойчивым национальным и международным признанием, и что его достижения признаны.

В частности эти доказательства демонстрируют, что:

\begin{enumerate}

    \item Алексей Инкин соответствует \underline{шести} из десяти критериев, перечисленных в 8
    CFR, разделе 204.5(h)(3), а именно:

    \begin{enumerate}[label=\roman*.]

        % 1
        \item Доказательства, что Алексей Инкин получал меньшие награды за профессиональное превосходство,
        признаваемые на национальном или международном уровне \SectionRef{subsec:Awards}.

        % 2
        \item Доказательства членства Алексея Инкина в ассоциациях в его профессиональной сфере,
        требующих выдающихся достижений для вступления \SectionRef{subsec:Associations}.

        \addtocounter{enumii}{1} % 3 -> 4
        % 4
        \item Доказательства, что Алексея Инкина приглашали оценивать работу других в составе комиссии
        \SectionRef{subsec:Judging}.

        % 5
        \item Доказательства значительного оригинального научного вклада Алексея Инкина
        в его сфере деятельности \SectionRef{subsec:Contributions}.

        \addtocounter{enumii}{2} % 6 -> 8
        % 8
        \item Доказательства выполнения Алексеем Инкиным критических ролей в выдающихся организациях
        \SectionRef{subsec:Role}.

        % 9
        \item Доказательства получения Алексеем Инкиным высокой зарплаты относительно других в его сфере
        \SectionRef{subsec:Salary}.

    \end{enumerate}

    \item Алексей Инкин также соответствует \ul{дополнительному критерию через сравнимые доказательства}.
    Его технические статьи отбирались различными командами редакторов
    для широкого распространения,
    и это сравнимо с критерием публикации научных статей для профессиональных учёных \SectionRef{subsec:Articles}.

    \item Алексей Инкин входит в малый процент поднявшихся на самый верх в области разработки ПО
    и обладает устойчивым международным признанием \SectionRef{sec:Merits}.

    \item Алексей Инкин принесёт существенную пользу США \SectionRef{sec:Benefit}.

\end{enumerate}

В соответствии с 8 CFR, разделом 204.5(h)(1),
Алексей Инкин может сам от своего имени подать петицию I-140 для классификации по
разделу 203(b)(1)(A) как негражданина с выдающимися способностями.

В соответствии с 8 CFR, разделом 204.5(h)(5),
для этой классификации не требуется ни приглашение от работодателя, ни разрешение на работу в США.

\pagebreak
