\subsubsection{Хакатон `Цифровой Прорыв', третье место, командное соревнование}
\label{subsubsec:AwardLod}

\SubSubSubSection{Контекст}

Алексей Инкин входил в команду `1DevFull', которая заняла третье место на хакатоне `Цифровой Прорыв'
20 июня 2021.
Команда соревновалась в кейсе для `iHerb', электронного магазина из США,
который был партнёром организаторов хакатона.
Всего было 6 кейсов (категорий) \ExhibitRef{RsvNews}.

Ключевая роль Алексея Инкина в команде и значимость награды продемонстрированы ниже.


\SubSubSubSection{Доказательство третьего места команды `1DevFull'}

Третье место команды `1DevFull' подтверждается анонсом от организаторов \ExhibitRef{RsvNews}.

Кроме того, оно было объявлено в Telegram-канале соревнования \ExhibitRef{LodTelegramPodium}.


\SubSubSubSection{Доказательство, что Алексей Инкин -- получатель награды}

Призовое место Алексея Инкина подтверждается электронным дипломом,
выданным организатором соревнования \ExhibitRef{LodDiploma}.

В этом дипломе не упоминается место, которое Алексей Инкин и его команда получили.
Его имя не упоминается в официальных публичных записях, потому что сайт хакатона
был недавно переделан и все анонсы старых наград были утрачены.

Вместо этого запись стрима с презентацией решений командой
может служить доказательством \ExhibitRef{LodDefenseTeam}.
Этот стрим выложен на официальном YouTube-канале, и это подтверждает его аутентичность.
Алексей Инкин показан на слайде в составе команды в роли `Архитектура + Машинное обучение',
а это ключевая позиция.

Кроме того, когда старые анонсы были удалены,
\MrLod
лично подтвердил награждение Алексея Инкина, написав рекомендацию
на его странице в LinkedIn \ExhibitRef{LodRecommendation}.

Тот факт, что \MrLod занимал должность <\dots>, подтверждается
этим анонсом:

\ParagraphQuoteBy{%
    <\dots>%
}{Новости на сайте <\dots>}


\SubSubSubSection{Доказательство, что награда за превосходство в разработке ПО}

В своей рекомендации на LinkedIn \MrLod упоминает:

\ParagraphByExhibit{%
    ``Критериями оценки были:

    \begin{itemize}
        \item \ul{программный прототип};
        \item решение главной проблемы держателя кейса;
        \item \ul{разработка и реализация};
        \item инновации;
        \item \ul{пользовательский интерфейс и удобство использования};
        \item решение второстепенных проблем;
        \item аналитика."
    \end{itemize}
}{\MrLod}{LodRecommendation}

Из этих критериев по крайней мере три (первый, третий и пятый) напрямую являются аспектами разработки ПО,
а остальные являются сопутствующими.
Это доказывает, что награда Алексея Инкина относится к разработке ПО.

Этот фокус соревнования также виден из анонса:

\ParagraphQuoteByExhibit{%
    На протяжении 48 часов 165 команд занимались решением кейсов Health \& Science,
    направленных на \ul{разработку ИТ-решений} для медицины и здравоохранения.
    Всего участниками было предложено 124 решения,
    40 команд проходят в финал конкурса.
    \ul{Анализ данных, цифровая трансформация, Big Data, машинное обучение,
    Data Science, искусственный интеллект, Web, Mobile} —
    этим и другим задачам были посвящены кейсы хакатона.%
}{Новость на сайте \QRsv}{RsvNews}


\SubSubSubSection{Доказательство федерального уровня награды}

Из публикации на сайте Министерства науки и вышего образования РФ:

\ParagraphQuoteByExhibit{%
    Участие в конкурсе могут принять команды IT-специалистов, дизайнеров и управленцев в сфере цифровой экономики
    старше 18 лет.%
}{Министерство науки и высшего образования РФ}{MinistryOfScience}

Кроме того, в соответствии этим анонсом, в пределах России не было географических ограничений на участие:

\ParagraphQuoteByExhibit{%
    `\dots Но посмотрите на результаты:
    \ul{4455 участников, 12 региональных площадок}, 6 кейсовых заданий, 124 решения.
    Я очень рад, что вас так много.
    Этот полуфинал действительно получился масштабным и, надеюсь, продуктивным и интересным для вас.

    «Отрадно, что в хакатоне участвовали не только айтишники,
    но и медики, и у некоторых из них есть опыт создания IT-стартапов.
    \ul{Формат конкурса объединяет людей абсолютно из разных уголков России}
    позволяя образовывать им сильные цифровые команды
    и создавая сильные межрегиональные связи в рамках ИТ-сообщества.
    Мы увидели много команд, участники которых живут в разных регионах нашей необъятной страны»,
    — сказал Алексей Агафонов, заместитель генерального директора АНО «Россия — страна возможностей»'%
}{Новость на сайте \QRsv}{RsvNews}


\SubSubSubSection{Доказательство национального признания награды}

Хакатоны `Цифровой Прорыв' анонсировались на сайте
Министерства науки и высшего образования РФ:

\ParagraphQuoteByExhibit{%
    Всероссийский конкурс «Цифровой прорыв» –
    один из \ul{флагманских проектов президентской платформы «Россия – страна возможностей»}.
    Он входит в план мероприятий Года науки и технологий.%
}{Министерство науки и высшего образования РФ}{MinistryOfScience}

С сайта организатора:

\ParagraphQuoteByExhibit{%
    Завершился третий хакатон третьего сезона Всероссийского конкурса «Цифровой прорыв» —
    \ul{флагманского проекта президентской платформы} «Россия — страна возможностей»,
    посвящённый теме «Медицина, здравоохранение, наука».%
}{Новость на сайте \QRsv}{RsvNews}

Организатор соревнования -- \QRsv,
некоммерческая организация, основанная по указу Президента России:

\ParagraphByExhibit{%
    ``в целях создания условий для повышения социальной мобильности,
    обеспечения личностной и профессиональной самореализации граждан постановляю:

    \begin{enumerate}

        \item Считать целесообразным создание автономной некоммерческой организации
        "Россия - страна возможностей" (далее - некоммерческая организация).

        \item Установить, что:

        \begin{enumerate}[label=\alph*)]
            \item учредителем некоммерческой организации от имени Российской Федерации
            является Управление делами Президента Российской Федерации;\dots"
        \end{enumerate}
    \end{enumerate}
}{\MrPutinT}{RsvDecree}

Вместе это означает, что награда признана на национальном уровне.
