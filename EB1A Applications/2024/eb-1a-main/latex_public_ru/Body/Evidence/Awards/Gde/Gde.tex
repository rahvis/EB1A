\subsubsection{Награда Google Developer Expert}
\label{subsubsec:AwardGde}

\SubSubSubSection{Доказательство, что Алексей Инкин -- получатель награды}

Алексей Инкин награждён званием Google Developer Expert в 2023 году.
Это можно проверить в его профиле на сайте Google.
На одном из значков в профиле написано:

\ParagraphQuoteExhibit{%
    Был награждён званием Google Developer Expert в 2023 году.%
}{GdeBadgeAwarded}

Эта награда присуждается одновременно с членством в программе Google Developer Experts,
как говорится в описании значка:
\QuoteExhibit{Получите этот значок, став Google Developer Expert.}{GdeBadgeDetails}.
Рекомендательное письмо Алексею Инкину от Google подтверждает текущее членство
и поэтому дополнительно подтверждает получение награды:

\ParagraphQuoteByExhibit{%
    Он состоит в сообществе Google Developer Experts (GDE)\dots
    Алексей Инкин является одним из них с 27 июля 2023.%
}{\MrGoogleT}{LetterGoogle}


\SubSubSubSection{Доказательство, что критерий присуждения награды -- профессиональное превосходство}

На главной странице программы Google Developer Experts перечислены критерии:

\ParagraphByExhibit{%
    \begin{itemize}%
%
        \item ``\ul{Твёрдый опыт} в области, для которой существует технология Google, например: Android,
        Google Cloud, Machine Learning, Web и другие.
        Формальное образование для рассмотрения заявки в GDE не нужно.

        \item Показать \ul{значительный вклад в сообщество разработчиков}, включая,
        но не ограничиваясь выступлениями на мероприятиях, публикацией контента,
        менторством других разработчиков и компаний."

    \end{itemize}
}{\GdeHome}{GdeHome}

Это доказывает, что основа для присуждения награды -- профессиональное превосходство.


\SubSubSubSection{Доказательство отсутствия ограничений для соискателей, кроме возраста 18+}

На той же главной странице есть полные критении:

\ParagraphByExhibit{%
    ``Эксперты GDE приходят \ul{из разных жизненных путей}.
    Это разработчики, основатели компаний, матери, активисты, и многие другие.
    Их объединяет то, что они страстные профессионалы
    с опытом в технологии Google, которые с удовольствием изучают новое,
    делятся опытом и влияют на сообщество.

    Критерии для подачи заявки:

    \begin{itemize}

        \item Твёрдый опытв области, для которой существует технология Google, например: Android,
        Google Cloud, Machine Learning, Web и другие.
        \ul{Формальное образование для рассмотрения заявки в GDE не нужно}.

        \item Показать значительный вклад в сообщество разработчиков, включая,
        но не ограничиваясь выступлениями на мероприятиях, публикацией контента,
        менторством других разработчиков и компаний."

        \item Способность понятно изъясняться и давать полезные советы другим.

        \item \ul{Возраст от 18 лет}.

        \item Способность пройти собеседование и общаться на английском, поскольку это официальный язык
        программы."

    \end{itemize}
}{Главная страница Google Developer Experts}{GdeHome}

Здесь указаны разные жизненные пути и отсутствие требований к образованию,
а единственное ограничение -- это возраст 18+.

Это ещё больше повышает значимость награды, поскольку почти все разработчики могут за неё конкурировать.

\SubSubSubSection{Доказательство международного признания награды}

То же самое рекомендательное письмо от Google показывает значимость
и международный статус награды, а также её редкость:

\ParagraphQuoteByExhibit{%
    Программа GDE -- это всемирное сообщество \ul{экспертов в технологии с большим опытом,
    которые являются лидерами мнений с опытом в технологиях Google,
    активными лидерами в этих сферах}, менторами от природы,
    и вносят вклад в широкие экосистемы разработчиков и стартапов.
    Почти тысяча Экспертов представляют технологии Google \ul{по всему миру},
    и из них \ul{около 105 экспертов специализируются на Dart и Flutter}.
    Алексей Инкин является одним из них с 27 июля 2023.%
}{\MrGoogleT}{LetterGoogle}

Для понимания, Flutter это технология, созданная Google для кросс-платформенной разработки приложений для веба, мобильных устройств и компьютеров.

Значимость награды дополнительно подтверждается привилегиями, которые она даёт:

\ParagraphByExhibit{%
    ``Google поддерживает Экспертов многими способами. Среди них:

    \begin{itemize}

        \item публичное признание их навыков на сайте Experts и с помощью значков,
        которые Эксперты могут размещать в социальных сетях и на своих сайтах,

        \item предоставление доступа к продуктовым командам и проектам Google,
        включая ранний доступ,

        \item приглашения на конференции Google по всему миру,

        \item \ul{оплата дороги} на конференции, где эксперты рассказывают о своих любимых технологиях."

    \end{itemize}
}{\GdeApplicationGuide}{GdeApplicationGuide}

Далее, статус Google Developer Expert признаётся и за пределами Google другими лидерами индустрии.
Например, JetBrains, второй в мире по популярности производитель профессиональных
редакторов кода для программистов, предоставляет много проуктов бесплатно для Экспертов программы
\ExhibitRef{GdeJetBrains}.

Далее, у страницы Google Developer Experts в LinkedIn
в три раза больше подписчиков, чем у Академии Кинематографических Искусств и Наук
(которые вручают Окаров), что показывает значительный профессиональный интерес
к программе по всему миру \ExhibitRef{GdeOscarsFollowers}.

Наконец, экспертов в программе по Flutter очень мало, лишь около 105 по всему миру
\ExhibitRef{LetterGoogle}, в то время как технология Flutter популярна:

\ParagraphQuoteByExhibit{%
    Мы продолжаем наблюдать быстрый рост использования Flutter,
    \ul{больше двух миллионов человек пробовали использовать Flutter} за первые полгода с выпуска этой технологии.
    Несмотря на беспрецедентную ситуацию в мире,
    в марте мы увидели рост 10\% к прошлому месяцу,
    и \ul{почти полмиллиона разработчиков используют Flutter} каждый месяц.%
}{\MrSneathT}{Flutter2M}

Эта награда помещает Алексея Инкина:

\begin{itemize}
    \item в приблизительно топ 0.025\% разработчиков, которые используют Flutter каждый месяц.
    \item в топ 0.00525\% тех, кто пробовал эту технологию.
\end{itemize}
