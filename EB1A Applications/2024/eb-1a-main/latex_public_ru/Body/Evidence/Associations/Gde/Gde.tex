\subsubsection{Членство в Google Developer Experts (GDE)}
\label{subsubsec:AssociationsGde}

Звание Алексея Инкина `Google Developer Expert'
уже упоминалось в другом критерии ранее \SectionRef{subsubsec:AwardGde}.
Однако, помимо награды, это ещё и престижное членство.

\SubSubSubSection{Доказательство членства Алексея Инкина в сообществе Google Developer Experts}

Членство Алексея Инкина в GDE подтверждается его публичным профилем на сайте Google.
Среди других значков есть
\QuoteExhibit{Член Google Developer Experts}{GdeBadgeAwarded}

Ещё одно доказательство его текущего членства -- рекомендательное письмо от Google:

\ParagraphQuoteByExhibit{%
    <\dots>
    Он член сообщества Google Developer Experts (GDE)\dots
    Алексей Инкин является одним из них с 27 июля 2023 года.%
}{\MrGoogleT}{LetterGoogle}


\SubSubSubSection{Доказательство, что GDE -- это сообщество в области Алексея Инкина -- разработке ПО}

То же самое рекомендательное письмо от Google делает акцент на том, что это сообщество:

\ParagraphQuoteByExhibit{%
    Программа GDE -- это всемирное сообщество \ul{экспертов в технологии} с большим опытом,
    которые являются лидерами мнений \ul{с опытом в технологиях Google},
    активными лидерами в этих сферах, менторами от природы,
    и \ul{вносят вклад в широкие экосистемы разработчиков и стартапов}.%
}{\MrGoogleT}{LetterGoogle}

Наконец, слово \Quote{Developer} в `Google Developer Experts' показывает, что это сообщество
относится к области Алексея Инкина -- разработке ПО.


\SubSubSubSection{Доказательство, что членство в GDE требует выдающихся достижений}

На главной странице сообщества указаны критерии вступления, и они содержат следующее:

\ParagraphByExhibit{%
    ``Критерии:

    \begin{itemize}

        \item ``\ul{Твёрдый опыт в области, для которой существует технология Google}, например: Android,
        Google Cloud, Machine Learning, Web и другие.
        Формальное образование для рассмотрения заявки в GDE не нужно.

        \item Показать \ul{значительный вклад} в сообщество разработчиков, включая,
        но не ограничиваясь выступлениями на мероприятиях, публикацией контента,
        менторством других разработчиков и компаний."

    \end{itemize}
}{\GdeHome}{GdeHome}

Слово \Quote{выдающиеся} не используется напрямую, но такой уровень требований
очевиден в следствие малого количества экспертов в сообществе.

На главной странице сообщества написано:

\ParagraphQuoteByExhibit{%
    Глобальное сообщество из более чем 1,000 профессионалов.%
}{\GdeHome}{GdeHome}

И рекомендательное письмо от Google содержит:

\ParagraphQuoteByExhibit{%
    \ul{Почти тысяча Экспертов} представляют технологии Google \ul{по всему миру},
    и из них \ul{около экспертов 105 специализируются на Dart и Flutter}.
    Алексей Инкин является одним из них с 27 июля 2023.%
}{\MrGoogleT}{LetterGoogle}

Это доказывает, что слово \Quote{выдающиеся} может быть использовано для достижений,
которые требуются для членства.

Для контекста: Dart это язык программирования, который используется в технологии Flutter,
они часто встречаются вместе в этой петиции.


\SubSubSubSection{%
    Доказательство, что кандидаты в GDE оцениваются
    экспертами в своих сферах, признанными на национальном и международном уровнях
}

Руководство по подаче заявки в GDE описывает процесс подачи заявки и её этапы её оценки.
Оно объясняет следующие этапы:

\ParagraphQuoteByExhibit{%
    Процесс вступления в программу Экспертов состоит из трёх шагов:
    форма заявления, собеседование с членом сообщества и продуктовое собеседование.

    \dots

    Если вы пройдёте этап формы, вас свяжут с
    \ul{одним из текущих GDE} для собеседования в сообществе.
    Цель этого собеседования -- узнать вас, вашу причину вступить
    в программу и ваш опыт в категории, на которую вы подаётесь.%
}{\GdeApplicationGuide}{GdeApplicationGuide}

Это доказывает, что существующие международно признанные эксперты оценивают кандидатов,
как и требует программа EB-1A.

На этом этапе Алексея Инкина оценивал \MrGdeMemberInterviewer,
как сказано в приглашении на собеседование \ExhibitRef{GdeMemberInterviewerInvitation}.
Он Google Developer Expert
и старший разработчик в <\dots>.
Его опыт указан на его странице в LinkedIn \ExhibitRef{GdeMemberInterviewerLinkedIn}.

Наконец, после этой стадии идёт ещё более тщательная проверка знаний и опыта:

\ParagraphQuoteByExhibit{%
    После собеседования с членом сообщества
    вас \ul{соединят с сотрудником Google} для продуктового собеседования.
    Это собеседование будет сфокусировано в основном на технических знаниях
    и опыте.%
}{\GdeApplicationGuide}{GdeApplicationGuide}

На этом этапе Алексея Инкина оценивал \MrGdeGoogleInterviewer,
как сказано в приглашении на собеседование \ExhibitRef{GdeGoogleInterviewerInvitation}.
Он инженер Google.
Его опыт указан на его странице в LinkedIn \ExhibitRef{GdeGoogleInterviewerLinkedIn}.
Он также является публичной фигурой, и у него более чем <много> подписчиков в Twitter \ExhibitRef{GdeGoogleInterviewerTwitter}.

Разработчики Google, конечно же, являются признанными экспертами в технологиях, которые они создают,
и это ещё больше доказывает факт, что кандидаты оцениваются международно признанными экспертами.
