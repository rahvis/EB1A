\subsubsection{PHPStan}
\label{subsubsec:ContributionsPhpStan}

\SubSubSubSection{Context}

PHP -- популярный язык программирования, который используют более 19\% всех программистов в мире \ExhibitRef{PhpStackOverflow}.

Это язык, на котором пишут бэкэнд, и в этой сфере он доминирует:

\ParagraphQuoteByExhibit{%
    PHP используется на 76.7\% всех сайтов, у которых известен язык бэкэнда.%
}{W3Tech}{PhpUsage}

Он используется на таких крупных сайтах как Facebook.com, Wikipedia.org, Vimeo.com и др. \ExhibitRef{PhpWebsites}.

У PHP нет встроенных инструментов для проверки кода до того, как программа запустится.
Поэтому появились сторонние инструменты для этого.
Один из них называется PHPStan.
Он

\ParagraphQuoteByExhibit{%
    \dots находит ошибки в вашем коде без написания тестов. Он бесплатный, и код открытый.%
}{Главная страница PHPStan}{PhpStanHome}

Алексей Инкин сделал значительную доработку этого инструмента, и это описано ниже.


\SubSubSubSection{Доказательство, что Алексей Инкин доработал PHPStan}

Конкретные пул-реквесты (правки кода) Алексея Инкина показаны в \ExhibitRef{PhpStanPrs}.
Их суть объясняется ниже.


\SubSubSubSection{Проблема и суть доработки PHPStan Алексеем Инкиным}

Проблема и суть доработки PHPStan Алексеем Инкиным описаны
в письме \MrPhpOne, <Компания и должность>:

\ParagraphQuoteByExhibit{%
    В 2020 году в этом инструменте не хватало важного функционала.
    Простыми словами -- этот инструмент \ul{требовал избыточных комментариев} в коде
    для корректного обнаружения некоторых типов ошибок.
    \ul{Эти комментарии не были нужны ни для чего другого, не помогали основному функционалу программы
    и требовали сил на поддержание в актуальном виде.
    Поэтому некоторые программисты не использовали их вовсе,
    и их код был подвержен некоторому классу ошибок},
    которые можно было обнаружить только при тестировании программы вручную.
    Эта проблема затрагивала и мою работу тоже.

    Техническими словами -- недостающий функционал -- это возможность объединения комментариев с документацией в подклассах
    с объединением и переопределением тегов.
    До доработки Алексея
    любой документационный комментарий в подклассе полностью заменял унаследованный от базового класса,
    и информация о типах аргументов и возвращаемых значений, не продублированных в новом комментарии, терялась.
    Это было критично, потому что в то время документационные комментарии
    были единственным способом расширить тип аргумента или сузить тип возвращаемого значения в PHP,
    и это до сих пор единственный способ уточнить тип элементов в массиве.
    Если программисту нужно было поменять один из этих типов в подклассе,
    им приходилось дублировать все остальные типы в новом комментарии, копируя комментарий из базового класса,
    чтобы PHPStan мог продолжать обнаруживать ошибки типов.

    Алексей Инкин разработал и внёс возможность объединения таких комментариев,
    сохраняя информацию о типах.

    В результате этой доработки:

    \begin{enumerate}

        \item \ul{Не нужно дублировать код} комментариев только для того, чтобы проверка типов продолжала работать.

        \item \ul{Нет ложных сообщений об ошибках}, если два комментария случайно рассинхронизируются.

        \item Поведение инструмента стало соответствовать тому, что программисты интуитивно ожидают,
        поэтому им \ul{не нужно тратить время на изучение неинтуитивного поведения}.

    \end{enumerate}

    Как я уже упоминал, PHPStan это \ul{самый популярный} статический анализатор кода на PHP,
    и его скачали \ul{более ста миллионов раз} с тех пор, как Алексей Инкин сделал эту доработку.
    \ul{Можно с уверенностью сказать, что этот функционал широко используется в индустрии в компаниях всех размеров}.
}{\MrPhpOneT}{LetterPhpOne}

Место работы и должность \MrPhpOne подтверждаются в \ExhibitRef{PhpOneCompanyRole}
и на его странице в LinkedIn \ExhibitRef{PhpOneLinkedIn}.

Системообразующая роль <компании> для языка PHP и сообщества разработчиков описана
на сайте компании:

\ParagraphQuoteByExhibit{%
    <\dots>%
}{<Сайт компании>}{PhpOneCompanyAbout}

Проблема была не только в том, что анализатор требовал избыточных комментариев для правильной работы.
Как утверждает <должность>,

\ParagraphQuoteByExhibit{%
    Для сохранения информации о типах программистам нужно было полностью копировать документационный комментарий
    из оригинального метода и вносить в него правки.

    Это было неудобно и трудно в поддержке, поэтому \ul{программисты часто не делали это.
    Вместо этого они просто отключали проверку типов, чтобы избежать ложных сообщений об ошибках
    в коде, если он уточнял тип в подклассе.
    Отключение проверки ведёт к возможным ошибкам.
    Даже большие проекты с миллионами строк кода страдали от этого}.

    Алексей Инкин исправил это.
    Он реализовал объединение документационных комментариев при переопределении метода.
    Теги ‘return’ и ‘param’ в переопределённом методе теперь заменяют информацию из базового класса, а теги ‘throws’ аккумулируются.
    Это избавило от необходимости дублировать комментарии в коде,
    и это объединение соответствует тому, что программисты интуитивно ожидают, когда пишут такой комментарий.
    \ul{Это позволило применять PHPStan в крупных проектах} без отключения проверки типов,
    что привело к \ul{повышению качества кода и стабильности проектов в индустрии в целом}.
    \ul{Использование PHPStan значительно выросло с мая 2020, когда это было реализовано}.%
}{\MrPhpTwoT}{PhpTwoRecommendation}

Должность \MrPhpTwo подтверждается его страницей в LinkedIn \ExhibitRef{PhpTwoLinkedIn}
и анонсом на сайте <компания> \ExhibitRef{PhpTwoCompanyRole}.

`<Должность>' является важной,
потому что <объяснение, которое содержится в> \ExhibitRef{PhpTwoCompanyAbout}.

Указанная проблема в PHPStan была настолько серьёзной и долгоживущей,
что о ней независимо сообщали 11 раз, хотя программысты перед сообщением об ошибке обычно проверяют,
было ли уже сообщение о такой ошибке.
См \ExhibitRef{PhpStanIssues} для полного списка отчётов об ошибках,
которые исправлены этой доработкой.
Обратите внимание, что в конце дискуссии по каждой из этихо ошибок основатель PHPStan пишет

\ParagraphQuoteByExhibit{%
    Шикарная работа!%
}{\MrMirtesT}{PhpStanIssues}

про личную работу Алексея Инкина.

Факт, что \MrMirtes является основателем PHPStan,
подтверждается на его странице в LinkedIn \ExhibitRef{MirtesLinkedIn}.


\SubSubSubSection{Доказательство широкого применения PHPStan}

Пакеты PHP обычно распространяются через сайт, который называется Packagist \ExhibitRef{ComposerWikipedia}.
Он отслеживает статистику скачиваний каждого пакета.
Он показывает, что PHPStan был скачан больше 131 миллиона раз
с момента доработки Алексеем Инкиным \ExhibitRef{PhpStanPackagist}.

Ещё одна важная метрика -- количество звёзд на GitHub.
У самого PHP 36 тысяч звёзд, а у PHPStan -- 12 тысяч \ExhibitRef{PhpStanStars}.
Если у вспомогательного инструмента всего в три раза меньше звёзд, чем у самого языка, которому он служит,
это означает, что инструмент очень популярен и важен.
См. \ExhibitRef{GitHubStars} о том, почему звёзды GitHub -- это применимая для оценки популярности метрика.
