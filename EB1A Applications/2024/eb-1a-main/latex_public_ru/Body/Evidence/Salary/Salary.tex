\subsection{%
    Доказательства получения Алексеем Инкиным высокой зарплаты относительно других в его сфере%
}
\label{subsec:Salary}

Алексей Инкин работал на Akvelon-Georgia LLC на должности `Software Development Engineer'
\ExhibitRef{AkvelonContract}.

Его первый день -- 6 июня 2022 года, подтверждаемый договором \ExhibitRef{AkvelonContract}.
Его последний день -- 16 июня 2023 года, подтверждаемый соглашением о расторжении договора \ExhibitRef{AkvelonTermination}.


\subsubsection{Средняя зарплата разработчиков ПО в грузии с 3 квартала 2022 по 2 квартал 2023}


\SubSubSubSection{По данным Forbes Georgia}

Forbes Georgia указывает, что медианная зарплата в IT в Грузии составляла 4500 лари в месяц в 2022 году
\ExhibitRef{ForbesSalary}.
Этому можно доверять, поскольку Forbes Georgia это официальное региональное издание
международного журнала Forbes \ExhibitRef{ForbesWikipedia}.

Это даёт годовую зарплату

\[
    4500\:GEL \cdot 12 = 54000\:GEL
\]

\SubSubSubSection{По данным Национальной службы статистики Грузии}

За этот промежуток времени Национальная служба статистики Грузии
показывает среднюю месячную зарплату до вычета налогов для профессий
`Компьютерное программирование, консультирование и связанные профессии'
(это самое близкое, что они смогли выдать к разработке ПО),
в своей справке \ExhibitRef{GeoStatResponse}:

\begin{center}
    \begin{tabular}{|l|r|r|r|r|}
        \hline
        & 2022 Q3 & 2022 Q4 & 2023 Q1 & 2023 Q2 \\
        \hline
        Месячная зарплата, GEL & 6870.10 & 6654.40 & 6391.70 & 6826.80 \\
        \hline
    \end{tabular}
\end{center}

Это даёт такую годовую зарплату с 3 квартала 2022 по 2 квартал 2023:

\[
    6870.10 \cdot 3 + 6654.40 \cdot 3 + 6391.70 \cdot 3 + 6826.80 \cdot 3 = 80229\:GEL
\]

Для справки: 1 доллар сша был равен 2.6118 лари (GEL) 16 июня 2023
\ExhibitRef{AkvelonBankRate},
в последний день работы Алексея Инкина в Akvelon.
Таким образом, средняя годовая зарплата разработчика ПО в грузии была

\[
    \frac{80229\:GEL}{2.6118\:\frac{GEL}{USD}} \approx 30717.90\:USD
\]


\subsubsection{Зарплата Алексея Инкина за 3 квартал 2022 -- 2 квартал 2023}

Договор Алексея Инкина с Akvelon установил ежемесячную зарплату в 3500 USD на первые три месяца
и 4000 USD после \ExhibitRef{AkvelonContract}, выплачиваемые в лари.
Эта зарплата потом была ещё увеличена.

В ходе своей работы в Akvelon, Алексей Инкин получил 155887.03 лари зарплаты после вычета налогов на свой счёт в банке,
что подтверждается выпиской \ExhibitRef{BogStatement}.

Зарплата до вычета налогов составила 167014.90 лари,
что подтверждается ответом Налоговой службы Грузии на запрос \ExhibitRef{AkvelonRsSalary}.

Алексей Инкин работал в Akvelon 1 год и 11 дней, или 1.0306 года,
поэтому эквивалентная годовая зарплата:

\[
    \frac{167014.90\:GEL}{1.0306} \approx 162055.98\:GEL
\]

Для справки: это

\[
    \frac{162055.98\:GEL}{2.6118\:\frac{GEL}{USD}} \approx 62047.62\:USD
\]


\subsubsection{Превышение зарплаты Алексея Инкина над средней}

\SubSubSubSection{Над данными Forbes}

Зарплата Алексея Инкина была в 3.001 раз выше, чем медианная зарплата в IT в 2022 году:

\[
    \frac{162055.98\:GEL}{54000\:GEL} \approx 3.001
\]

\SubSubSubSection{Над данными Национальной службы статистики Грузии}

Зарплата Алексея Инкина была в 2.02 раза выше средней зарплаты для должностей
`Компьютерное программирование, консультирование и связанные профессии' за ровно тот же промежуток с 3 квартала 2022 по 2 квартал 2023 Q2:

\[
    \frac{162055.98\:GEL}{80229\:GEL} \approx 2.02
\]

\pagebreak
