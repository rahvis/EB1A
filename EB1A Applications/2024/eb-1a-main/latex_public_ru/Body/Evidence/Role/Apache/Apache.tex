\subsubsection{\Asf, Apache Beam}
\label{subsubsec:RoleApache}

\SubSubSubSection{Контекст}

\Asf — это большая некоммерческая организация, занимающаяся разработкой ПО с открытым исходным кодом.

`Apache Beam` — один из её значимых проектов.
Он используется для пакетной обработки данных.

Выдающаяся репутация как фонда, так и проекта Beam, а также роль Алексея Инкина в проекте будут продемонстрированы ниже.

\SubSubSubSection{Доказательство работы Алексея Инкина на проекте Apache Beam}

Поскольку Apache Software Foundation является некоммерческой организацией, работающей за счёт волонтёров,
её разработчики не заключают договоры.
Единственное требование для участия — это подписание одностороннего соглашения,
как сказано на странице \ExhibitRef{ApacheContributorAgreements},
и Алексей Инкин сделал это \ExhibitRef{ApacheIclaSigned}, \ExhibitRef{ApacheIclaAccepted}.

Вместо двустороннего договора можно непосредственно продемонстрировать код, написанный Алексеем Инкиным.

Главная страница Apache Beam показана на \ExhibitRef{BeamHome}.
На ней есть кнопка `Link to GitHub Repo`, которая ведёт на репозиторий кода на GitHub,
и этот скриншот также подтверждает адрес репозитория.

Код находится на GitHub, см. \ExhibitRef{GitHubWikipedia} для пояснения, что такое GitHub.

Конкретные пул-реквесты Алексея Инкина (правки кода) показаны в \ExhibitRef{BeamPrs}.
Их суть будет показана ниже.

Наконец, в справке по GitHub объясняется, что такое пул-реквесты и почему они означают вклад \ExhibitRef{Prs}.


\SubSubSubSection{Доказательство того, что Алексей Инкин выполнял критическую роль в проекте Apache Beam}

Из письма от \MrApacheT:

\ParagraphQuoteByExhibit{%
    С 2022 по середину 2023 года
    сообщество Beam разрабатывало два веб-приложения, чтобы упростить обучение Apache Beam,
    позволяя пользователям писать код онлайн и без установки запускать его на серверах в "песочнице":

    \begin{itemize}

        \item Apache Beam Playground — это веб-приложение, чтобы запускать код и экспериментировать с ним.

        \item Tour of Beam — это интерактивный учебник с темами, собранными в дерево,
        отслеживающий прогресс пользователя и включающий функционал Playground
        для экспериментов и запуска примеров кода по каждой теме.

    \end{itemize}

    \ul{Алексей Инкин внес вклад в оба эти приложения} с июня 2022 по июнь 2023 года.
    Я не работал напрямую с господином Инкиным над этими проектами;
    Я знаком с этими приложениями, потому что <\dots>.
    Господин Инкин работал напрямую с другими членами сообщества Beam и PMC Beam.
    \ul{Его работа включала проектирование архитектуры, авторство 43 изменений кода,
    а также лидирующую роль -- проверку 47 изменений кода от трёх других фронтенд-разработчиков}.

    \ul{Вклад господина Инкина был основополагающим для завершения этих проектов}
    в срок к ежегодной конференции Beam Summit 2023 года,
    которая является главным мероприятием проекта, где мы делаем анонсы и демо,
    и \ul{в ней приняли участие 592 человека}, включая онлайн-участников.
    \ul{С тех пор этими двумя приложениями воспользовались более 3000 человек}.%
}{\MrApacheT}{LetterApache}

<Должность> подтверждает, что работа Алексея Инкина была основополагающей
для своевременного завершения проекта.


\SubSubSubSection{Доказательство того, что \MrApache является <Должность>}

Страница <\dots> на сайте \Asf указывает \MrApache как <Должность>
\ExhibitRef{ApacheRoles}


\SubSubSubSection{Доказательство того, что \Asf является выдающейся организацией}

\Asf был основан в 1999 году \ExhibitRef{ApacheWikipedia}.

\Asf настолько известен, что на странице в Wikipedia об открытом исходном коде
\Asf упоминается первым примером формальной организации в open-source:

\ParagraphQuoteByExhibit{%
    Многие крупные формальные организации были созданы для поддержки развития
    движения open-source, включая Apache Software Foundation,
    которая поддерживает проекты, такие как фреймворк Apache Hadoop и HTTP-сервер Apache HTTP.%
}{Страница в Wikipedia об open-source}{OpenSourceWikipedia}

Как сказано в ежегодном отчёте Фонда за 2023 финансовый год,
среди его спонсоров такие компании, как
Microsoft, Google, Yahoo!, Facebook, IBM, Visa и другие
\ExhibitRef{ApacheSponsors}.

У страницы \Asf в LinkedIn больше 69 тысяч подписчиков,
и там следующие цифры:

\ParagraphQuoteByExhibit{%
    Более 850 индивидуальных членов и более 8200 коммитеров успешно сотрудничают,
    создавая свободно доступное ПО промышленного уровня, которым пользуются миллионы людей по всему миру:
    тысячи программных решений распространяются под лицензией Apache.%
}{Страница \Asf в LinkedIn}{ApacheLinkedIn}

Это делает \Asf выдающейся организацией для целей этой петиции.


\SubSubSubSection{Доказательство того, что Apache Beam является выдающимся проектом внутри \Asf}

\Asf поддерживает множество проектов.
Популярность проектов обычно измеряют количеством звёзд на GitHub.

В справке по GitHub говорится:

\ParagraphQuoteByExhibit{%
    Звёзды, которые вы ставите, позволяют легче найти репозиторий или тему позже\dots
    Звёзды также означают признательность автору за его работу.
    Многие рейтинги репозиториев на GitHub основаны на количестве звёзд.
    Кроме того, Explore GitHub показывает популярные репозитории по количеству звёзд.%
}{Справка по GitHub}{GitHubStars}

Это доказывает, что количество звёзд является показателем популярности проекта.

Аккаунт \Asf в GitHub содержит 85 страниц репозиториев.
Если отсортировать их по количеству звёзд, репозиторий Beam будет на первой странице
\ExhibitRef{BeamGitHubStars}.
Это делает его популярнее 98.8\% других репозиториев Apache.

Ещё одним важным показателем является то, что Beam — один из репозиториев в \Asf
с наибольшим количеством коммитов, как показано в ежегодном отчёте фонда \ExhibitRef{BeamCommits}.

Этот показатель также важен для популярности проекта, потому что коммит — это

\ParagraphQuoteByExhibit{%
    как снимок вашего репозитория.
    Коммиты это снимки всего вашего репозитория в определённое время.
    Рекомендуется делать новые коммиты часто, основываясь на логических единицах изменений.
    Со временем коммиты должны рассказывать историю развития вашего репозитория
    и то, как он стал таким, каким он является сейчас.%
}{Справка по GitHub}{GitCommit}

Наконец, проект Apache Beam был начат компанией Google, а затем передан
в \Asf для дальнейшей разработки \ExhibitRef{BeamWikipedia}.

Google уделяет много внимания Apache Beam и создаёт свои продукты на его основе.
Один из них называется `Dataflow` \ExhibitRef{DataflowDocs}.

Это делает Apache Beam выдающимся проектом в рамках выдающейся организации
для целей этой петиции.

Кроме того, на главной странице документации Dataflow есть прямые ссылки
на `Tour of Beam` и `Beam Playground` \ExhibitRef{DataflowDocs} — два приложения, в разработке фронтенда которых Алексей Инкин был руководителем команд фронтенда.
Такая значимость этих приложений дополнительно подчеркивает
важность роли Алексея Инкина в \Asf и Apache Beam.
