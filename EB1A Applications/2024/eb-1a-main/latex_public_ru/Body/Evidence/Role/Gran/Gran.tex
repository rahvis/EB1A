\subsubsection{\GSchool}
\label{subsubsec:RoleGran}

\SubSubSubSection{Контекст}

Школа Граня -- производитель образовательных раборов для детей
и сеть развивающих центров для детей с некоторыми программами для взрослых.

Алексей Инкин был техническим директором компании в 2015--2016, и это будет показано ниже.


\SubSubSubSection{Доказательство, что Алексей Инкин выполнял критическую роль в бизнесе Сергея Граня}

Сергей Грань, основатель и владелец Школы Граня, пишет следующее в своём письме:

\ParagraphQuoteByExhibit{%
    Я работал с Алексеем Инкиным с 2011 года, когда он стал моим помощником в задачах, связанных с IT.
    С 2014 он стал техническим директором компании и оставался им
    до своего ухода в 2016. Его обязанности включали:

    \begin{enumerate}

        \item Создание сайтов, включая программирование бэкенда.

        \item Интеграцию внешних информационных систем: платежи, отслеживание рекламы, системы Email рассылки,
        сервис организации партнерских программ и т.п.

        \item Программирование систем управления обучением.

        \item Найм других программистов и других фрилансеров и управление их работой.

    \end{enumerate}

    Основным вкладом Алексея в работу моей компании было то, что \ul{он полностью самостоятельно решал все вопросы, связанные с IT-обеспечением} нашей работы.

    В результате работы Алексея в нашей компании были внедрены различные эффективные IT-решения
    в областях: организации обучения клиентов, организации продаж, приема различных видов платежей, организации рассылок,
    организации партнерской программы и другие,
    что позволило нашей компании \ul{в несколько раз увеличить скорость расширения
    и помогло увеличить в 4 раза доходы компании} за период с 2014 по 2015 год.

    На пике работы в моей компании было \ul{больше 100 преподавателей, больше 2500 учеников одновременно
    и более 100 активных партнеров}.

    Этот объём стал возможным в том числе благодаря эффективным IT-решениям,
    внедренным Алексеем в разные области работы компании.%
}{\MrGranT}{LetterGran}


\SubSubSubSection{Доказательство, что Школа Граня -- выдающаяся организация}

У видео на канале Сергея Граня в YouTube, продвигающем его школу,
больше 1.2 миллиона просмотров в сумме \ExhibitRef{GranYouTube}.

На страницу его школы на VK.com, популярной социальной сети в России,
подписаны более 16 тысяч человек, и на ней рассказывается о продуктах школы \ExhibitRef{GranSchoolVk}.

KTRK, национальное телевидение Киргизии, выпустило сюжет
о развивающем центре для детей в Бишкеке, столице Киргизии,
с фокусом на методологии Сергея Граня и образовательных наборах,
которые его школа производит.
См. скриншоты и транскрипт в \ExhibitRef{ZamanaVideo}.
Популярность этого канала подтверждается гайдом BBC про СМИ Киргизии
\ExhibitRef{BbcUtrk}.

Центральное телевидение Челябинской области России
проводило дискуссию об инженерном образовании для детей и подростков,
и один из франчайзи Сергея Граня выступал спикером
на одной панели с такими видными деятелями образования
как Виталий Литке, заместитель министра образования и науки Челябинской области
\ExhibitRef{ChelyabinskTv}.

В этом видео показаны образовательные наборы, выпускаемые Школой Граня,
с логотипом школы, чётко видимым на двух из этих наборов.

Значительное количество подписчиков и сюжеты в СМИ в двух странах
делают Школу Граня выдащейся организацией для целей этой петиции.
