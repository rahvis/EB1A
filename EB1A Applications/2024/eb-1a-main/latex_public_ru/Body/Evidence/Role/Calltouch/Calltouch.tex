\subsubsection{Calltouch}
\label{subsubsec:RoleCalltouch}

\SubSubSubSection{Context}

Calltouch -- сервис интернет-маркетинга в России.

Выдающаяся репутация компании и роль Алексея Инкина в ней
будут показаны ниже.

\SubSubSubSection{Доказательство работы Алексея Инкина в Calltouch в прошлом}

Алексей Инкин работал в Calltouch с 29 июля 2019 года по 18 сентября 2020 года.
Факт работы и дата начала подтверждаются его трудовым договором с Calltouch \ExhibitRef{CalltouchContract}.


\SubSubSubSection{Доказательство выполнения Алексеем Инкиным критической в Calltouch}

Вот цитата из письма \MrCalltouchT:

\ParagraphQuoteByExhibit{%
    В ходе своей работы Алексей сделал \ul{значимый вклад} в наши продукты:

    \textbf{Исправление критических уязвимостей}.
    Мы делали нашу систему аутентификации более надёжной и гибкой,
    и нам нужен был аудит кода, который работает с привилегиями.
    Алексей сделал его как часть своей работы.
    \ul{Он нашёл и починил несколько критических уязвимостей в коде до того, как они были выкачены
    и могли быть использованы.
    Это нельзя недооценивать}.

    \textbf{Рефакторинг системы аутентификации}.
    Кроме исправлений, Алексей довёл до конца миграцию на более продвинутую систему
    управления привилегиями и помог отключить старую систему.
    Это \ul{освободило большое количество ресурсов, занятых её поддержанием},
    и уже даже это повысило производительность отдела.

    \textbf{Инструмент визуализации дерева привилегий}.
    Побочным эффектом миграции системы аутентификации было дублирование привилегий
    и избыточность в наследовании ролей.
    Изначально мы могли видеть только текстовое представление этого наследования в БД.
    Алексей разработал инструмент, который визуализирует это дерево,
    чтобы легко обнаруживать избыточность, чтобы от неё можно было избавляться.
    Это \ul{уменьшило ментальные ресурсы на поддержание системы}.

    \textbf{Доработка PHPStan}.
    Чтобы контролировать качество кода, наша компания полагалась на ревью, тесты
    и встроенные инструменты IDE (редакторов кода).
    Мы не использовали статический анализ в наших процессах,
    потому что стандартный анализатор PHPStan на нашем коде
    показывал много ложных несуществующих ошибок, связанных с типами переменных.
    В апреле и мае 2020, \ul{Алексей использовал своё свободное время для значительной доработки этого инструмента.
    Это сделало возможным его применение в наших процессах.
    Он помог интегрировать этот инструмент в наш деплой.
    Эта дополнительная проверка значительно уменьшила частоту сбоев
    и частоту необходимых экстренных фиксов.
    Это повысило качество продуктов и улучшило пользовательский опыт.
    Это было критически важно, потому что это было во время пандемии,
    и клиенты быстро уходили при сбоях в работе сервиса}.%
}{\MrCalltouchT}{LetterCalltouch}

<Должность> определяет роль Алексея Инкина в компании как критическую,
приводя много подтверждений этому.

В соответствии с политикой USCIS,

\ParagraphQuoteByExhibit{%
    Примеры лидирующей или критической роли могут включать, но не ограничиваться:
    \dots
    критической ролью для выдающейся организации или выдающегося подразделения
    компании с детальным разъяснением директора
    или главного исследователя соответствующей организации ли подразделения.%
}{\PolicyManual}{RoleUscisPolicy}

Это делает роль Алексея Инкина в Calltouch критической для целей этой петиции.


\SubSubSubSection{Доказательство, что \MrCalltouch -- <Должность>}

<Страница> показывает, что \MrCalltouch -- <Должность>
\ExhibitRef{CalltouchRole}.

Дополнительно эта должность указана на странице \MrCalltouch's в LinkedIn \ExhibitRef{CalltouchLinkedIn}.


\SubSubSubSection{Доказательство, что Calltouch -- выдающаяся организация}

Доход Calltouch за 2019 год упоминается в газете `Ведомости':

\ParagraphQuoteByExhibit{%
    По итогам 2019 года выручка Calltouch — порядка 700 млн руб.%
}{Газета `Ведомости'}{CalltouchSoldVedomosti}

Центральный банк России показывает, что
61.9057 российских рублей были равны 1 доллару США в последний день 2019 года
\ExhibitRef{CalltouchCbr}.

Получается, что доход Calltouch за 2019 год -- приблизительно

\[
    \frac{700\:million\:RUB}{61.9057\:\frac{RUB}{USD}} \approx 11.3\:million\:USD
\]

В этой же статье говорится:

\ParagraphQuoteByExhibit{%
    \dots это крупнейшая M\&A-сделка с участием сервиса аналитики маркетинга на российском рынке.%
}{Газета `Ведомости'}{CalltouchSoldVedomosti}

Газета `Ведомости' -- это авторитетный источник такой информации, что подтверждается статьёй о ней в Wikipedia:

\ParagraphQuoteByExhibit{%
    Ведомости основаны в 1999 году как совместное предприятие Dow Jones,
    который публикует \ul{The Wall Street Journal}; Pearson, который раньше публиковал \ul{Financial Times};
    и Independent Media, которые публикуют The Moscow Times.%
}{Статья в Wikipedia про Ведомости}{VedomostiWikipedia}

Такой большой доход, упомянутый в таком авторитетном издании, делает Calltouch выдающейся организацией
для целей этой петиции.
