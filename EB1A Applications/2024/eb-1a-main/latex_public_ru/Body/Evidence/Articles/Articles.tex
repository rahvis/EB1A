\subsection{%
    Сопоставимое доказательство: Доказательство, что технические статьи Алексея Инкина были отобраны для широкого распространения%
}
\label{subsec:Articles}

Это доказательство сопоставимо с критерием vi
`Авторство научных статей в профессиональных изданиях или других крупных изданиях'.

\subsubsection{%
    Доказательство, что критерий научных статей не совсем применим к Алексею Инкину%
}

Алексей Инкин работает в индустрии, а не в научной среде.
Политика USCIS напрямую приводит это как пример не совсем применимого критерия:

\ParagraphQuoteByExhibit{%
    Например, если \ul{публикация научных статей
    не совсем применима к человеку с профессией в индустрии, а не в научной среде},
    петиционер может продемонстрировать, что презентация работы
    на крупной торговой выставке сопоставима по значимости с этим критерием.%
}{\PolicyManual}{ComparableUscisPolicy}


\subsubsection{%
    Контекст
}

С ноября 2021 года Алексей Инкин пишет и публикует технические статьи в своём блоге на Medium
-- популярной площадке блогов:\\
https://medium.com/@alexey.inkin

Список статей Алексея Инкина показан в \ExhibitRef{MediumArticles}.
В сумме у них больше 97 тысяч просмотров и больше 45 тысяч прочтений.

Его статьи отбирались для широкого распространения как минимум в трёх случаях:

\begin{itemize}

    \item На самом Medium, внутренней командой редакторов платформы в первой половине 2022 года.

    \item На самом Medium, внутренней командой редакторов платформы для их обновлённой программы `Boost' в декабре 2022 года.

    \item В блоге компании `Flutter Devs' с 95 тысячами подписчиков в LinkedIn в ноябре 2023.

\end{itemize}

Всё это объясняется и доказывается ниже.


\subsubsection{%
    Отбор для распространения на Medium в первой половине 2022 года%
}

До середины 2022 года у Medium была программа, где их команда редакторов отбирала качественные статьи
для широкого распространения тем, кто не подписан на автора.

Из 10 статей, опубликованных Алексеем Инкиным к тому времени,
8 были отобраны командой редакторов Medium.

Рекомендации для отбора были изложены на их справочной странице того времени:

\ParagraphByExhibit{%
    ``Тысячи статей публикуются на Medium каждый день.
    Наша цель -- донести лучше из них до читателей Medium.
    Мы делаем это с помощью команды редакторов, которые просматривают недавно опубликованные статьи
    и выбирают из них соответствующие высоким стандартам.

    Что значит, когда статья отобрана?
    Когда статья отобрана, она помечается подходящей для распространения среди читателей Medium во всех местах —
    на главной странице, на тематических страницах, в приложении, в ежедневной подборке и в почтовых рассылках —
    и распространяется рекомендательной системой Medium\dots

    На что редакторы обращают внимание в статьях

    Мы ценим качественный контент — свежие идеи, уникальный взгляд, различные голоса,
    умное мышление — и мы верим, что читатели тоже.
    Вот на что редакторы обращают внимание, оценивая статью:

    \begin{itemize}

        \item Соответствует ли статья высоким редакторским стандартам? –
        Хорошо ли она написана, легко ли читается, нет ли ошибок, есть ли ссылки на источники, чёткий ли нарратив, убедительно ли?

        \item Несёт ли она ценность для читателя? –
        Содержит ли она новые открытия и взгял?
        Предлагает ли новый оригинальный подход к привычному?
        Пробуждает ли эмоции и заставляет ли думать?
        Даёт ли ценные советы?
        Обогащает ли понимание читателем темы?
        Есть ли ощущение, что время проведено не зря?

        \item Написана ли она для читателя? –
        Написана ли статья с заботой о читателе?
        Может ли читатель связать статью с собой или с какой-то важной темой?

        \item Закончена ли статья? –
        Является ли текст законченным и выверенным?
        Разумно написанным?
        Получит ли читатель удовлетворение?

        \item Тверды ли утверждения? –
        Обоснованы ли утверждения?
        Есть ли ссылки на источники упомянутых фактов?
        Выдержит ли статья проверку?

        \item Честная ли она? –
        Написана ли статья искренне?
        Правдивая ли?

        \item Удобно ли её читать? –
        Нормально ли она отформатирована для веба и мобильных устройств?
        Заголовок подходящий и понятный, чтобы узнать, что в ней?
        Читаемый ли сам текст — абзацы, отступы, стили, разделы, цитаты?

        \item Не грязная ли?
        Нет ли опечаток и ошибок?

        \item Подходящие ли картинки?
        Соответствуют ли они статье?"

    \end{itemize}
}{Medium Help Center}{MediumHelpCuration}

Сам Medium нигде не упоминал, как распознать, что статья была отобрана редакторами.
Однако, популярные блоги об издании статей поясняют, что признак этого --
точка на приватном графике статистики и надпись \Quote{Distributed} рядом с ней
\ExhibitRef{MediumDistributedExplained}.

Чтобы доказать, что статьи Алексея Инкина были отобраны, можно обратиться к приватным скриншотам статистики.
Графики показывают, что 8 из 10 статей, написанных Алексеем Инкиным к середине 2022 года, имеют этот знак \ExhibitRef{MediumStatsDistributed}.


\subsubsection{%
    Отбор для новой программы `Boost' на Medium%
}

В конце 2022 и начале 2023 года,
Medium запустил новую программу `Boost' вместо старой системы отбора,
как объяснил их CEO:

\ParagraphQuoteByExhibit{%
    \textbf{Новый Boost для топовых статей}

    \dots У нас всегда было много способов поднять статью на Medium,
    включая наш алгоритм рекомендаций, теги, рассылки и издания.

    Сегодны мы запускаем значительно более сильный Boost\dots

    Три фактора определяют процесс:

    \begin{itemize}

        \item Мы полагаемся на редакторов изданий, которые будут давать нам рекомендации.
        Они -- наше сообщество редакторов.

        \item Живые люди в Medium -- внутренняя команда редакторов --
        подтверждаем эти рекомендации в соответствии с нашими стандартами распространения.

        \item Наши алгоритмы сопоставляют эти качественные статьи
        с читателями, которым они будут наиболее интересны,
        и приоритезируют эти статьи для широкого распространения на Medium.

    \end{itemize}

    Эти три фактора заслуживают более подробного объяснения.

    \textbf{Редакторы из сообщества}

    \dots Сейчас мы работаем с редакторами 15 изданий,
    которые находят статьи и рекомендуют их для Boost.
    Скоро мы включим ещё 15 изданий\dots
}{\MrStubblebineT}{MediumBoostOriginal}

Статья Алексея Инкина `Never have separate sign-in routes'
опубликована в сентябре 2022 и была отобрана для буста в декабре 2022 года,
что видно по значку `Boosted' на странице её статистики.
Значок гласит:

\ParagraphQuoteByExhibit{%
    Поздравляем!
    Наша команда редакторов выбрала вашу статью для широкого распространения 15 декабря 2022 года.%
}{Приватная страница статистики}{SignInRoutesStats}

Полный текст статьи приведён в \ExhibitRef{SignInRoutes}.

Вот как можно грубо оценить, сколько статей получают Boost ежемесячно.

Менеджер продукта в Medium писал в июне 2023 года:

\ParagraphQuoteByExhibit{%
    Мы расширили программу с 15 до 62 номинаторов за последние несколько месяцев.%
}{\MrsStallingsT}{StallingsFaq}

\MrMcCarty, один из номинаторов, рассказал, сколько статей им разрешают номинировать:

\ParagraphQuoteByExhibit{%
    Нас выбрали для участия в пилотной программе на Medium,
    которая позволяет нам номинировать 5 статей в неделю для возможного буста редакторами Medium.%
}{\MrMcCarty, номинатор Medium}{MediumNominationCount}

Его роль номинатора упоминается в том же самом посте \MrsStallingsT
\ExhibitRef{StallingsFaq}, внизу скриншота.

Если в месяце в среднем 4.35 недели, то получается 1349 номинаций в месяц:

\[
    62 \cdot 5 \cdot 4.35 \approx 1349
\]

В том же посте от июня 2023 \MrsStallings пишет:

\ParagraphQuoteByExhibit{%
    Наш долгосрочный план -- чтобы большинство номинаций исходило из сообщества.
    Но пока, в мае 2023, \ul{только 43\% исходят из него}.%
}{\MrsStallingsT}{StallingsFaq}

Это означает, что с учётом внутренних номинаций от самих редакторов Medium
общее количество номинаций в месяц -- около

\[
    \frac{1349}{43\%} \approx 3137
\]

BusinessInsider.com пишет,
что около 20 тысяч статей публиковались на Medium каждый день в 2018 году \ExhibitRef{BusinessInsider}.

В 2023 году по данным директора Medium

\ParagraphQuoteByExhibit{%
    Десятки тысяч статей публикуются на Medium каждый день.%
}{\MrStubblebineT}{StateOfMedium2023}

По самой низкой оценке это даёт 600 тысяч статей в месяц:

\[
    20,000 \cdot 30 = 600,000
\]

Даже если все номинации одобряют для буста, верхняя граница процента статей, получающих буст:

\[
    \frac{3,137}{600,000} \approx 0.52\%
\]

Этот процент должен быть ещё меньше, потому что на кажодм шаге мы использовали худшие оценки:

\begin{itemize}
    \item 20,000 статей в день это наименьшая оценка `десятков тысяч'.
    \item Не все из 1349 номинаций в месяц используются.
    \item Не все номинации одобряются.
\end{itemize}

Ещё важное значение имеет то, что статья Алексея Инкина получила Boost в декабре 2022,
в то время как программу с номинациями из сообщества анонсировали в феврале 2023.
Это означает, что программа тогда была экспериментальной, и, скорее всего, номинаций в ней было меньше.

Всё это делает выбор статьи Алексея Инкина его выдающимся достижением.

Также примечательно, что у статьи больше тысячи аплодисментов (это термин для лайков на Medium) \ExhibitRef{SignInRoutes}.
Для сравнения: официальные статьи по Flutter на Medium получают количество аплодисментов того же порядка (тысячи),
см. примеры в \ExhibitRef{FlutterArticleClaps}.
Если чьи-то статьи по Flutter получают лайки того же порядка,
что и официальные анонсы самой технологии, это должно быть значительным.


\subsubsection{Отбор для блога FlutterDevs в LinkedIn}

Ещё одна статья Алексея Инкина, `The new lint in Dart 3.2',
была опубликована на странице компании FlutterDevs в LinkedIn \ExhibitRef{FlutterDevsPost}.

Начало статьи приведено в \ExhibitRef{NewLintDart32}.

Как указано на их сайте, FlutterDevs это частная компания,
которая разрабатывает приложения на Flutter \ExhibitRef{FlutterDevsHome}.

Примечательно, что на их страницу в LinkedIn подписано больше 95 тысяч человек \ExhibitRef{FlutterDevsPost}.
Это больше, чем у большинства топовых компаний и групп в LinkedIn по запросу \Quote{flutter}
\ExhibitRef{FlutterSearchLinkedIn}.

Как видно из ленты,
FlutterDevs публикует несколько постов в неделю на своей странице в LinkedIn \ExhibitRef{FlutterDevsPosts},
что делает публикацию там редкой привилегией.


\subsubsection{%
    Доказательство количества просмотров и прочтений%
}

Скриншоты приватных страниц статистики каждой статьи приведены в
\ExhibitRef{SignInRoutesStats}, \ExhibitRef{MediumStatsDistributed} и \ExhibitRef{MediumStatsRegular}.

Они суммированы в таблице в \ExhibitRef{MediumStats}, и в результате получается больше 97 тысяч просмотров и больше 45 тысяч прочтений.

\pagebreak
