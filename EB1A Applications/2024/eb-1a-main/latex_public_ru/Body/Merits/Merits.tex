\section{Окончательное определение выдающихся способностей Алексея Инкина}
\label{sec:Merits}

Страсть Алексея Инкина к разработке ПО стала проявляться ещё в школе.
Он делал бесплатные компьютерные игры как хобби ещё в подростковом возрасте.
В 2002 году, когда ему было 17 лет, у его игр были сотни скачиваний в месяц из интернета
\ExhibitRef{Lehasoft}.

Его игры публиковались в сборниках на компакт-дисках в районе 2003 года минимум дважды
без его ведома \ExhibitRef{Cds}.

В возрасте 18 лет Алексей Инкин начал свой первый бизнес в сфере ПО.
Он создал каталог GetSoft.ru для русскоязычной аудитории в 2004 году \ExhibitRef{GetSoft}.
На пике сайт посещали 40 тысяч уникальных посетителей в месяц \ExhibitRef{GetSoftStat}.

В 2005--2006 Алексей Инкин работал на полставки в Telma -- международной компании,
специализирующейся на ПО для мобильных телефонов.

С 2002 по 2006 Алексей Инкин учился в Нижегородском Государственном Техническом университете и получил диплом бакалавра с отличием \ExhibitRef{Diploma1}.
С 2006 по 2008 он продолжил обучение там и получил диплом инженера с отличием.
Все оценки в его дипломе -- `отлично' \ExhibitRef{Diploma2}.

С 2006 по 2013 Алексей Инкин делал сайты как фрилансер.

В 2011 он стал помощником Сергея Граня и вскоре стал техническим директором
в его образовательном центре, что принесло успех, как Сергей описывает это:

\ParagraphQuoteByExhibit{%
    Основным вкладом Алексея в работу моей компании было то, что он полностью самостоятельно решал все вопросы, связанные с IT-обеспечением нашей работы.

    В результате работы Алексея в нашей компании были внедрены различные эффективные IT-решения
    в областях: \ul{организации обучения клиентов, организации продаж, приема различных видов платежей, организации рассылок,
    организации партнерской программы и другие},
    что позволило нашей компании в несколько раз увеличить скорость расширения
    и помогло \ul{увеличить в 4 раза доходы компании} за период с 2014 по 2015 год.

    На пике работы в моей компании было \ul{больше 100 преподавателей, больше 2500 учеников одновременно
    и более 100 активных партнеров}.

    Этот объём стал возможным в том числе благодаря эффективным IT-решениям,
    внедренным Алексеем в разные области работы компании.%
}{\MrGranT}{LetterGran}

В 2019--2020 Алексей Инкин работал Старшим разработчиком PHP в Calltouch,
крупной платформе рекламной аналитики в России.
Он выполнял критическую рооль, отвечая за систему аутентификации.

\ParagraphQuoteByExhibit{%
    В ходе своей работы Алексей сделал значимый вклад в наши продукты:

    \ul{Исправление критических уязвимостей}.
    Мы делали нашу систему аутентификации более надёжной и гибкой,
    и нам нужен был аудит кода, который работает с привилегиями.
    Алексей сделал его как часть своей работы.
    \ul{Он нашёл и починил несколько критических уязвимостей в коде до того, как они были выкачены
    и могли быть использованы.
    Это нельзя недооценивать}.

    Рефакторинг системы аутентификации.
    Кроме исправлений, Алексей довёл до конца \ul{миграцию на более продвинутую систему
    управления привилегиями} и помог отключить старую систему.
    Это освободило большое количество ресурсов, занятых её поддержанием,
    и уже даже это \ul{повысило производительность отдела}.%
}{\MrCalltouchT}{LetterCalltouch}

Также в 2020 Алексей Инкин внёс значимый вклад в индустрию.
Он использовал своё свободное время, чтобы доработать самый популярный в мире статический анализатор кода на языке PHP --
PHPStan.
Это значительно улучшило результаты анализа, исключив много ложных сообщений об ошибках.
Важность этого объяснил <должность> в Calltouch:

\ParagraphQuoteByExhibit{%
    Чтобы контролировать качество кода, наша компания полагалась на ревью, тесты
    и встроенные инструменты IDE (редакторов кода).
    Мы не использовали статический анализ в наших процессах,
    потому что стандартный анализатор PHPStan на нашем коде
    показывал много ложных несуществующих ошибок, связанных с типами переменных.
    В апреле и мае 2020, Алексей использовал своё свободное время для значительной доработки этого инструмента.
    Это сделало возможным его применение в наших процессах.
    Он помог интегрировать этот инструмент в наш деплой.
    Эта дополнительная проверка \ul{значительно уменьшила частоту сбоев
    и частоту необходимых экстренных фиксов}.
    Это повысило качество продуктов и улучшило пользовательский опыт.
    Это было критически важно, потому что это было во время пандемии,
    и клиенты быстро уходили при сбоях в работе сервиса.%
}{\MrCalltouchT}{LetterCalltouch}

Эта доработка произвела эффект по всей индустрии.
Популярность PHPStan сильно выросла с тех пор,
\ul{код с этой правкой был скачан более 130 миллионов раз} \ExhibitRef{PhpStanPackagist}

Это неудивительно, потому что PHP -- это преобладающий язык для бэкэнда,
более 76\% сайтов в мире используют PHP \ExhibitRef{PhpUsage}.

Это подтверждает \MrPhpOne, <должность>:

\ParagraphQuoteByExhibit{%
    Можно с уверенностью сказать, что этот функционал широко используется в индустрии в компаниях всех размеров.%
}{\MrPhpOneT}{LetterPhpOne}

Это также подтверждает \MrPhpTwoT:

\ParagraphQuoteByExhibit{%
    Это позволило применять PHPStan в крупных проектах без отключения проверки типов,
    что привело к повышению качества кода и стабильности проектов в индустрии в целом.%
}{\MrPhpTwoT}{PhpTwoRecommendation}

В конце 2020 Алексей Инкин начал заниматься Flutter -- фреймворком от Google для кросс-платформенной
разработки для веба, мобильных устройств и компьютеров.

В 2021 он принял участие в крупнейшем хакатоне в России -- `Цифровом Прорыве' --
в составе команды `1DevFull' в роли архитектора \ExhibitRef{LodDefenseTeam}.
Команда заняла третье место в своём кейсе \ExhibitRef{RsvNews}.

Также в 2021 Алексей Инкин начал вести свой блог на Medium.
На текущий момент он написал 49 технических статей
и несколько нетехнических
\ExhibitRef{MediumArticles}.

Его статьи несколько раз отбирались для широкого распространения.
8 статей были отобраны командой Medium для рекомендаций тем, кто не подписан на него \ExhibitRef{MediumStatsDistributed}.
Одна из его статей была выбрана для широкого буста на Medium \ExhibitRef{SignInRoutesStats}.
Ещё одна статья опубликована в специализированном блоге на LinkedIn с 95 тысячами подписчиков \ExhibitRef{FlutterDevsPost}.

На блог Алексея Инкина на Medium подписаны более 600 человек.
Это может выглядеть незначительно, но такова специфика этой платформы.
Для сравнения, на менеджера Google по продуктам Dart и Flutter,
который публикует официальные анонсы, там подписано чуть больше 6 тысяч человек
\ExhibitRef{ThomsenFollowers}.
Для технического писателя иметь всего в 10 раз меньше подписчиков,
чем у менеджера, поддерживающего эти технологии,
о которых он пишет, должно считаться успехом после всего двух лет написания статей.

В 2022--2023 Алексей Инкин работал в Akvelon -- международной компании со штаб-квартирой в США.
В Akvelon он выполнял критическую роль,
возглавляя фронтенд для приожений Apache Beam:

\ParagraphQuoteByExhibit{%
    <\dots>%
}{\MrAkvelonT}{LetterAkvelon}

Кроме работы непосредственно для Apache, Алексей Инкин возглавлял разработку
продукта, который остался в собственности у Akvelon:

\ParagraphQuoteByExhibit{%
    <\dots>%
}{\MrAkvelonT}{LetterAkvelon}

\MrEditorT говорит:

\ParagraphQuoteByExhibit{%
    <\dots>

    До того, как они выпустили редактор с этим функционалом, редакторы в приложениях Flutter закрывали только базовые потребности программистов,
    оставляя ощущение, что до настоящего профессионального десктоп-редактора далеко.
    В результате программисты часто писали большие фрагменты кода в других редакторах
    и потом вставляли его в приложение, где этот код был им нужен.

    Функционал, разработанный Алексеем Инкиным, \ul{поменял эту парадигму, теперь больше всего может быть сделано
    непосредственно в приложении на Flutter, и профессиональные десктопные редакторы нужны реже.
    Это имеет огромное значение для индустрии, поскольку упрощает работу,
    снижает зависимости от десктопа и делает большее количество задач решаемыми чисто на мобильных устройствах}.

    Мы уже видим это по широкому применению их редактора.
    Он стал вторым по популярности по статистике репозитория pub.dev.
    Он уже используется в чатах, онлайн-редакторах конфигурации, крупных онлайн-редакторах и т.п.%
}{\MrEditorT}{LetterEditor}

flutter\_code\_editor стал ключевым компонентом двух приложений,
которые Алексей Инкин и его команда под его руководством разработала для Apache,
одного из крупнейших в мире сообществ open-source.
Его роль подтверждается письмом от \MrApache:

\ParagraphQuoteByExhibit{%
    С 2022 по середину 2023 года
    сообщество Beam разрабатывало два веб-приложения, чтобы упростить обучение Apache Beam,
    позволяя пользователям писать код онлайн и без установки запускать его на серверах в "песочнице":

    Алексей Инкин внес вклад в оба эти приложения с июня 2022 по июнь 2023 года\dots
    Его работа включала \ul{проектирование архитектуры, авторство 43 изменений кода,
    а также лидирующую роль -- проверку 47 изменений кода от трёх других фронтенд-разработчиков}.

    Вклад господина Инкина был основополагающим для завершения этих проектов
    в срок к ежегодной конференции Beam Summit 2023 года,
    которая является главным мероприятием проекта, где мы делаем анонсы и демо,
    и в ней приняли участие 592 человека, включая онлайн-участников.
    С тех пор этими двумя приложениями воспользовались более 3000 человек.%
}{\MrApacheT}{LetterApache}

Независимо от Akvelon Алексей Инкин создал и поддерживает 23 open-source-пакета для Flutter,
включая популярные \ExhibitRef{PubAinkin}.

Он вносил правки в несколько чужих пакетов с популярностью 96\%
и один пакет, поддерживаемый Google, с популярностью 99\%
(более популярный, чем 99\% всех пакетов на языке Dart) \ExhibitRef{FlutterContributions}.

Его вклад в open-source настолько важен, что в сочетании с его экспертными знаниями
он принёс ему звание `Google Developer Expert' \ExhibitRef{GdeBadgeAwarded},
и вот как Google официально его характеризует:

\ParagraphQuoteByExhibit{%
    Он член сообщества Google Developer Experts (GDE).
    Программа GDE -- это всемирное сообщество экспертов в технологии с большим опытом,
    которые являются лидерами мнений с опытом в технологиях Google,
    активными лидерами в этих сферах, менторами от природы,
    и вносят вклад в широкие экосистемы разработчиков и стартапов.%
    Почти тысяча Экспертов представляют технологии Google по всему миру,
    и из них около 105 экспертов специализируются на Dart и Flutter.
    Алексей Инкин является одним из них с 27 июля 2023.%
}{\MrGoogleT}{LetterGoogle}

Несмотря на то, что награда свежая, она требовала долгой истории вклада.
Руководство по подаче заявки для вступления в программу GDE упоминает два года как ориентир:

\ParagraphQuoteByExhibit{%
    Убедитесь, что тщательно заполнили форму и продемонстрировали своё влияние и вклад
    за продолжительное время (\ul{в идеале не менее двух лет}).
    У успешных кандидатов в GDE долгая история технических выступлений и статей, написанных ими.%
}{Руководство по вступлению в GDE, Страница `Application Process'}{GdeApplicationGuide}

Это показывает, что у Алексея Инкина было устойчивое признание не менее двух лет к лету 2023.

Ещё более долгая история значимой работы требуется для звания старшего члена IEEE,
которое Алексей Инкин получил в сентябре 2023 \ExhibitRef{IeeeElevationNotification}.
Из требований:

\ParagraphQuoteByExhibit{%
    Кандидаты должны быть в профессии минимум 10 лет%

    Кандидаты должны продемонстрировать \ul{значимую работу на протяжении как минимум 5 из этих лет}.%
}{Страница с требованиями к старшим членам IEEE}{IeeeSeniorRequirements}

Это крупнейшая в мире техническая профессиональная организация \ExhibitRef{Samsung}.

Более того, Алексею Инкину было доверено оценивать заявки новых кандидатов на звание старшего члена IEEE
\ExhibitRef{IeeeJudgingThankYou}.

Также \Siia (SIIA) приняла Алексея Инкина в жюри премии CODiE 2024 года \ExhibitRef{CodieApproved}.
Эта организация и награда имеют долгую историю.
Организация создана в 1984 году как `\Spa'.
Она вручала награду `Excellence in Software Award' с 1986 года \ExhibitRef{SiiaWikipedia}.
Эту награду называли `Оскарами в софте' Los Angeles Times \ExhibitRef{CodieLatimes}
и Washington Post \ExhibitRef{CodieWp}.
Ассоциацию переименовали в \Siia после объединения с \Iia в 1999 году \ExhibitRef{SiiaWikipedia},
тогда их называли \QuoteByExhibit{двумя самыми влиятельными группами, связанными с компьютерами}{Wired}{SiiaWired},
и награду переименовали в CODiE \ExhibitRef{CodieRename}.

Эта история явно показывает экстраординарные способности Алексея Инкина.

Последние достижения и экспертная награда от Google,
помещающая его в \ul{топ 0.025\%} экспертов в своей технологии,
доказывают, что он поднялся на самый верх в области разработки ПО,
как требует 8 C.F.R. §204.5(h)(2).

Два года обязательных достижений для программы Google Developer Experts
и 5 лет значимой работы, требуемой для IEEE,
и назначение в жюри `Оскаров в софте',
доказывают, что он обладает устойчивым национальным и международным признанием
и что его достижения признаны в его сфере,
как требует by 8 C.F.R. §204.5(h)(3).

\pagebreak


%Lehasoft TankBattle:
%628 monthly
%https://web.archive.org/web/20030104004237/http://online.download.ru:80/program.ehtml?ProgramID=11268
%
%Lehasoft Dyna
%176 monthly
%https://web.archive.org/web/20030925125216/http://online.download.ru/program.ehtml?ProgramID=11592
%
%Lehasoft Dyna DM
%248 all-time by 2007
%https://web.archive.org/web/20070324093153/http://www.download.ru/soft/arcade/games__entertainment/Lehasoft_Dyna_DM/11592
%
%Lehasoft Tetris
%347 monthly
%https://web.archive.org/web/20021019072912/http://online.download.ru/program.ehtml?ProgramID=10100
%
%
%
%Русская игротека 2003
%https://gamestracker.org/torrents/pc/action/russkaja_igroteka_2003_400_igr_na_russkom_dlja_windows/15-1-0-37113
%
%
%Сборник игр для мальчиков
%http://www.torrentsbornik.ru/torrent/igry/161998_sbornik_igr_dlya_malchikov_piratka
