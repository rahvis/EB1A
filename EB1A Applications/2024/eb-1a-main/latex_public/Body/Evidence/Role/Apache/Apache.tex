\subsubsection{\Asf, Apache Beam}
\label{subsubsec:RoleApache}

\SubSubSubSection{Context}

\Asf is a large non-profit community that produces open-source software.

`Apache Beam' is one of their significant projects.
It is used for batch data processing.

The distinguished reputation of both the Foundation and Beam, as well as \mrls role
in the project will be shown below.

\SubSubSubSection{Proof of \mrls contribution to Apache Beam}

As Apache Software Foundation is a non-profit driven by volunteers,
the contributors to their projects are not under contracts.
The only thing required for contributing is unilaterally signing an agreement
as explained in \ExhibitRef{ApacheContributorAgreements},
which \mrl did \ExhibitRef{ApacheIclaSigned}, \ExhibitRef{ApacheIclaAccepted}.

Instead of a bilateral contract, direct code contributions can be demonstrated.

The home page for Apache Beam is shown on \ExhibitRef{BeamHome}.
It has the button `Link to GitHub Repo', which leads to the code repository on GitHub,
and that screenshot also verifies the address of that repository.

The code is on GitHub, see \ExhibitRef{GitHubWikipedia} on what GitHub is.

The specific pull requests (code contributions) of \mrl are shown in \ExhibitRef{BeamPrs}.
The nature of them will be explained below.

Finally, the help page on GitHub explains why those items are code contributions \ExhibitRef{Prs}.


\SubSubSubSection{Proof of \mrls performing a critical role in Apache Beam}

Here is the quote from a letter by \MrApacheT:

\ParagraphQuoteByExhibit{%
    From 2022 to mid-2023,
    the Beam community developed two web applications to simplify learning Apache Beam
    by allowing users to write their code online
    and execute it on sandboxed backend servers without any installation:

    \begin{itemize}

        \item Apache Beam Playground is a web app to experiment with code and run it.

        \item Tour of Beam is a learning tool with a tree of topics
        that tracks a user’s progress along those topics
        and also combines the Playground functionality to experiment and run code samples
        in each topic.

    \end{itemize}

    \ul{Mr. Alexey Inkin contributed to both of those applications} from June 2022 to June 2023.
    I did not work directly with Mr Inkin on these projects;
    I am familiar with these applications because <\dots>.
    Mr Inkin worked directly with other members of the Beam community and Beam PMC.
    \ul{His work included architecture design, authorship of 43 code changes,
    as well as a leading role, reviewing 47 code changes from three other frontend developers}.

    \ul{Mr. Inkin’s contribution was central to the completion of those projects}
    on time for the annual Beam Summit conference of 2023,
    which is the main event for the project where we make announcements and showcases,
    and \ul{it was attended by 592 people} including online participants.
    \ul{The two applications have now been used by over 3000 people}.%
}{\MrApacheT}{LetterApache}

<Title> witnesses that the work of \mrl was central
for the project completion on time.


\SubSubSubSection{Proof that Mr. \MrApache is a <Title>}

The <\dots> page on the website of \Asf lists \MrApache as <Title>
\ExhibitRef{ApacheRoles}


\SubSubSubSection{Proof that \Asf is a distinguished organization}

\Asf was established in 1999 \ExhibitRef{ApacheWikipedia}.

\Asf is so widely known that the Wikipedia page for `open source' lists
\Asf as the first example of a formal institution of open-source software:

\ParagraphQuoteByExhibit{%
    Many large formal institutions have sprung up to support the development
    of the open-source movement, including the Apache Software Foundation,
    which supports community projects such as the open-source framework
    Apache Hadoop and the open-source HTTP server Apache HTTP.%
}{Wikipedia page for `open source'}{OpenSourceWikipedia}

As shown in the Foundation's annual report for the fiscal year 2023,
its sponsors include
Microsoft, Google, Yahoo!, Facebook, IBM, Visa, and others
\ExhibitRef{ApacheSponsors}.

The LinkedIn page for \Asf has over 69 thousand followers
and goes into numbers:

\ParagraphQuoteByExhibit{%
    850+ individual Members and 8,200+ Committers successfully collaborate
    to develop freely available enterprise-grade software,
    benefiting millions of users worldwide:
    thousands of software solutions are distributed under the Apache License.%
}{\Asf LinkedIn Page}{ApacheLinkedIn}

This makes \Asf a distinguished organization for the purpose of this petition.


\SubSubSubSection{Proof that Apache Beam is a distinguished project within \Asf}

\Asf maintains a lot of projects.
A common way to measure the popularity of projects is the number of starts on GitHub.

From a GitHub help page,

\ParagraphQuoteByExhibit{%
    Starring makes it easy to find a repository or topic again later\dots
    Starring a repository also shows appreciation to the repository maintainer for their work.
    Many of GitHub's repository rankings depend on the number of stars a repository has.
    In addition, Explore GitHub shows popular repositories based on the number of stars they have.%
}{GitHub Help Page}{GitHubStars}

This proves that the number of stars is a valid metric for project popularity.

The GitHub account for \Asf has 85 pages of repositories.
When they are sorted by the star count, Beam repository is present on the first page
\ExhibitRef{BeamGitHubStars}
This makes it more popular than 98.8\% other repositories of Apache.

Another important metric is that Beam is one of the repositories in \Asf
with the highest number of commits as shown in the annual report of the Foundation \ExhibitRef{BeamCommits}.

This metric is valid for the popularity of a project because a commit is

\ParagraphQuoteByExhibit{%
    like a snapshot of your repository.
    These commits are snapshots of your entire repository at specific times.
    You should make new commits often, based around logical units of change.
    Over time, commits should tell a story of the history of your repository
    and how it came to be the way that it currently is.%
}{GitHub Help Page}{GitCommit}

Finally, the Apache Beam project was started by Google before being handed over
to the \Asf for further development \ExhibitRef{BeamWikipedia}.

Google pays a lot of attention to Apache Beam and has its products based on Apache Beam.
One of them is called `Dataflow' \ExhibitRef{DataflowDocs}.

This makes Apache Beam a distinguished project within a distinguished organization
for the purpose of this petition.

Additionally, the main page of Dataflow documentation has direct links
to `Tour of Beam' and `Beam Playground' \ExhibitRef{DataflowDocs}, the two applications for which \mrl
had been leading the development of frontend.
This prominence of those applications further highlights
the importance of \mrls role in \Asf and Apache Beam.
