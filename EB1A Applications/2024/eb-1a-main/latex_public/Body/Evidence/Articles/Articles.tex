\subsection{%
    Comparable Evidence: Evidence of \mrls technical articles being curated for wide online publication%
}
\label{subsec:Articles}

This evidence aims to be comparable to the criterion vi
`The person's authorship of scholarly articles in the field,
in professional or major trade publications or other major media'.

\subsubsection{%
    Proof that publication of scholarly articles is not readily applicable to \mrl%
}

\mrl has been working in the industry and not in academia.
In the USCIS Policy Manual, this case is explicitly given as an example
of a not readily applicable criterion:

\ParagraphQuoteByExhibit{%
    For instance, if \ul{the publication of scholarly articles
    is not readily applicable to a person whose occupation is in an industry rather than academia},
    a petitioner might demonstrate that the person's presentation of work
    at a major trade show is of comparable significance to that criterion.%
}{\PolicyManual}{ComparableUscisPolicy}


\subsubsection{%
    Context
}

Since November 2021, \mrl writes and publishes technical articles in his blog on Medium,
a popular blogging platform:\\
https://medium.com/@alexey.inkin

The list of all \mrls articles is shown in \ExhibitRef{MediumArticles}.
Combined, they have over 97 thousand views and over 45 thousand reads.

His articles were selected for wide distribution in at least 3 ways:

\begin{itemize}

    \item On Medium itself, by the curation team of the platform 2022 Q1-2.

    \item On Medium itself, by the curation team of the platform for their updated `Boost' program in December 2022.

    \item In `Flutter Devs' blog with 95 thousand followers on LinkedIn in November 2023.

\end{itemize}

All of that is explained and proven below.


\subsubsection{%
    Curation for distribution on Medium in 2022 Q1-2%
}

Until mid-2022, Medium had an old program with their curation team picking up quality articles
for distribution wider than authors' followers.

Out of the 10 articles published by \mrl by then,
8 were selected by the curation team at Medium.

The guidelines for curation were explained in their help center at the time:

\ParagraphByExhibit{%
    ``Thousands of stories are published every day on Medium.
    Our goal is to share the best of these stories with Medium readers.
    We do this with the help of our curation team, who reviews recently published stories
    and selects those that meet a high editorial standard for curation.

    What does it mean to be curated?
    When a story is curated, it becomes eligible to be distributed to readers across Medium surfaces —
    on the homepage, on topic pages, in our app, in our Daily Digest newsletter, and in other emails —
    and shared via Medium's recommendation system\dots

    What curators look for in a story

    We value quality content — fresh ideas, unique perspectives, varied voices,
    smart thinking — and believe readers do, too.
    Here are the elements our curation team considers in evaluating story quality:

    \begin{itemize}

        \item Does the story meet a high editorial standard? –
        Is it well-written, easy to follow, free of errors, appropriately sourced, narratively strong, and compelling?

        \item Does it add value for the reader? –
        Does it share new insights or perspectives?
        Offer an original take on a familiar issue?
        Does it stir emotions and/or thinking?
        Provide meaningful advice?
        Enrich a reader’s understanding of the topic?
        Does it feel like time well spent?

        \item Is it written for the reader? –
        Is the story written with the reader in mind?
        Does the story make a connection with the reader or to a larger issue?

        \item Is it complete? –
        Is it a finished, polished piece of work?
        Considered?
        Concise?
        Will a reader walk away satisfied?

        \item Is it rigorous? –
        Are claims supported?
        Sources cited alongside stated facts?
        Does the story hold up to scrutiny?

        \item Is it honest? –
        Is the story written in good faith?
        Is it truthful?

        \item Does it offer a good reading experience? –
        Is it properly formatted for the web/mobile?
        Does it have a clear and relevant headline that lets the reader know what the story is about?
        An easily readable story body — paragraphs/spacing/styling/section breaks/quotes?

        \item Is it clean?
        Is it free of typos and errors?

        \item Is the imagery appropriate?
        Is the imagery relevant and appropriate to the story?"

    \end{itemize}
}{Medium Help Center}{MediumHelpCuration}

Medium itself does not have a help page that shows what exactly the signs of a curated article are.
However, popular blogs on publishing explain that this can be seen
by a dot on a private statistics chart and a label \Quote{Distributed} near it
\ExhibitRef{MediumDistributedExplained}.

To prove that \mrls articles were curated, private screenshots of statistics can be used.
The charts of 8 of 10 articles by \mrl published by mid-2022 have that sign \ExhibitRef{MediumStatsDistributed}.


\subsubsection{%
    Curation for the new `Boost' program on Medium%
}

In late 2022 and early 2023,
Medium launched the new `Boost' program instead of the old one,
as explained by their CEO:

\ParagraphQuoteByExhibit{%
    \textbf{A new Boost for top stories}

    \dots We've always had many ways to boost a story on Medium,
    including our recommendation algorithm, tags, newsletters, and publications.

    What we are launching today is a much bigger Boost\dots

    There are three factors determining the process by which stories will get Boosted:

    \begin{itemize}

        \item We are relying on publication editors to send in suggestions.
        They are our community curators.

        \item Humans at Medium, aka our internal curation team,
        confirm these suggestions according to our distribution standards.

        \item Our algorithms match these high-quality story suggestions
        with the readers most likely to be interested in each story,
        and weight those stories for extra distribution across Medium.

    \end{itemize}

    These three factors deserve a bit more explanation.

    \textbf{Community curation}

    \dots We are currently working with 15 publication editors
    to find and submit story suggestions for the Boost.
    We will be onboarding 15 more publications\dots
}{\MrStubblebineT}{MediumBoostOriginal}

The article `Never have separate sign-in routes' by \mrl
was published in September 2022 and chosen for boost in December 2022,
which is seen by the `Boosted' badge on its statistics page.
The badge reads:

\ParagraphQuoteByExhibit{%
    Congratulations!
    Our curation team selected your story for further distribution on December 15th, 2022.%
}{Private statistics page}{SignInRoutesStats}

The full text of the article is in \ExhibitRef{SignInRoutes}.

To roughly estimate how many articles get boosted every month,
the following math can be applied.

A product manager at Medium wrote in June 2023:

\ParagraphQuoteByExhibit{%
    We've scaled the program from 15 to 62 nominators in the past few months.%
}{\MrsStallingsT}{StallingsFaq}

\MrMcCarty, one of the nominators, shares how many they are allowed to nominate:

\ParagraphQuoteByExhibit{%
    We have been chosen to participate in a pilot program on Medium
    that allows us to nominate five stories a week to be considered for boosting by the Medium curators.%
}{\MrMcCarty, a nominator for Medium}{MediumNominationCount}

The fact that he is a nominator is mentioned in the same post of \MrsStallingsT
\ExhibitRef{StallingsFaq}, bottom of the screenshot.

If a month contains 4.35 weeks on average, this makes around 1349 nominations per month:

\[
    62 \cdot 5 \cdot 4.35 \approx 1349
\]

According to \MrsStallings from the same post in June 2023,

\ParagraphQuoteByExhibit{%
    Our long-term plan is that the vast majority of nominations will come via our community nominators.
    However, as of May 2023, \ul{only 43\% do}.%
}{\MrsStallingsT}{StallingsFaq}

This means that with the nominations by the Medium internal team,
the total number of nominations is around

\[
    \frac{1349}{43\%} \approx 3137
\]

According to BusinessInsider.com,
about 20 thousand articles were being published on Medium every day in 2018 \ExhibitRef{BusinessInsider}.

In 2023, according to the CEO of Medium,

\ParagraphQuoteByExhibit{%
    There's tens of thousands of new stories published on Medium every day.%
}{\MrStubblebineT}{StateOfMedium2023}

With the lowest estimate, that gives about 600 thousand articles per month:

\[
    20,000 \cdot 30 = 600,000
\]

Even if all nominations for boost get approved, that would cap the percentage of boosted articles at

\[
    \frac{3,137}{600,000} \approx 0.52\%
\]

This percentage should be even smaller because the worst estimate was used at all steps:

\begin{itemize}
    \item 20,000 articles a day is the lowest estimate of `tens of thousands'.
    \item Not all of the possible 1349 Boost nominations per month are used.
    \item Not all Boost nominations are approved.
\end{itemize}

Another consideration is that the article of \mrl was boosted in December 2022,
while the program with community nominators was announced in February 2023.
This means that the program was experimental back then, and likely they had even less nominations.

All of that makes the selection of the article by \mrl an extraordinary achievement.

Also note that the article has over thousand claps (the Medium term for likes) \ExhibitRef{SignInRoutes}.
For comparison, the official Flutter articles on Medium have likes in the order of thousands,
see examples in \ExhibitRef{FlutterArticleClaps}.
If someone's article on Flutter has likes in the same order of magnitude
as the official announcements of the technology itself, this should be considered high.


\subsubsection{Curation for FlutterDevs' LinkedIn blog}

Another article of \mrl, `The new lint in Dart 3.2',
was posted on the LinkedIn page of FlutterDevs \ExhibitRef{FlutterDevsPost}.

The beginning of the article is given in \ExhibitRef{NewLintDart32}.

As seen from their website, FlutterDevs is a private company
that develops applications in Flutter \ExhibitRef{FlutterDevsHome}.

What is notable is that their LinkedIn page has over 95 thousand followers \ExhibitRef{FlutterDevsPost}.
This is more than most of the top companies and top groups on LinkedIn for \Quote{flutter}
\ExhibitRef{FlutterSearchLinkedIn}.

As seen from the list of their posts,
FlutterDevs publishes few posts per week on their LinkedIn page \ExhibitRef{FlutterDevsPosts},
which makes an article posted there a rare high privilege.


\subsubsection{%
    Proof of view and read count%
}

The screenshots of private statistics pages for each individual article are given in
\ExhibitRef{SignInRoutesStats}, \ExhibitRef{MediumStatsDistributed}, and \ExhibitRef{MediumStatsRegular}.

They are summarized in \ExhibitRef{MediumStats} to produce over 97 thousand views and over 45 thousand reads.

\pagebreak
